% Chapter Template
\providecommand{\rootfolder}{../..} % Relative path to main.tex
\documentclass[\rootfolder/main.tex]{subfiles}
\begin{document}

\chapter{Implementation of the behavior tree control system} % Main chapter title
\label{ch:implementation-control} % Change X to a consecutive number; for referencing this chapter elsewhere, use \ref{ch:foo}

To replace the existing state machine-based control system, a behavior tree control system was implemented.
This chapter will provide details on the goals of this system, the choices that were made in planning the system and a description of the implementation of the system itself.

%----------------------------------------------------------------------------------------
%	SECTION: GOALS
%----------------------------------------------------------------------------------------

\section{Goals}

At the outset of this thesis, the existing solution which used a state machine approach, was evaluated.
\acrshort{smach} is a task-level state machine architecture for the \acrfull{ros}.
There are some intrinsic issues with large state machines, in particular concerning maintainability and scalability.
As the cyborg state machine grew in size and complexity, these became increasingly apparent.

As outlined in \cref{ch:background}, the events that govern transitions between states in a state machine are tightly coupled.
Because of this, when adding or removing a state, it is necessary to update the transitions between this state and potentially every other state in the state machine.
Furthermore, a state machine with many states can be difficult to grasp as it becomes cluttered, and the many states and transitions require the user to maintain a complex mental model.
Reusability of behaviors can also be difficult, as the states are tightly coupled to their transitions.
Finally, as outlined in the \acrshort{smach} documentation, \acrshort{smach} is less well suited for unstructured tasks \cite{Bohren}.
As one goal of the NTNU Cyborg project is life-like behavior, it was found that a decision making approach more suited for unstructured behavior was desirable.

%----------------------------------------------------------------------------------------
%	SECTION: EVALUATION OF ALTERNATIVES
%----------------------------------------------------------------------------------------

\section{Evaluation of alternatives}

Before implementing a behavior tree control system for the Cyborg, several existing implementations and libraries were evaluated.
Among these implementations were two existing implementation in \arcshort{ros} called \emph{ROS-Behavior-Tree} \cite{Colledanchise} and \emph{pi\_trees} \cite{Goebel}.
Also, a behavior tree library not specifically for \acrshort{ros} is evaluated, called \emph{behavior3} by the author of \cite{Pereira2015}.
The main considerations in choosing an implementation are that it needs to be written in \CC or Python, as those are the available languages in \acrshort{ros}.
Furthermore, it would be beneficial if it were possible to specify the behavior tree in a configuration file.

\subsection{Pi Trees}

\begin{listing}
    \inputminted[fontsize=\scriptsize]{Python}{\rootfolder/Chapters/Chapter3/Listings/pitrees.py}
    \caption[Implementation of a behavior tree using Pi Trees]{Implementation of a behavior tree using Pi Trees, from the Pi Trees documentation.}
    \label{lst:pitrees}
\end{listing}

Pi Trees is implemented by the Pi Robot project.
The library is implemented in Python, and is designed specifically for use with \acrshort{ros} \cite{Goebel}.
Behavior trees are implemented in code, rather than being specified using a configuration file, which adds some complexity to the design process.
As an example, consider a simple tree implementation from the Pi Robot project \cite{PiRobot} shown in \cref{lst:pitrees}.

The Iterator node used here has not been discussed previously, as it is less commonly used.
It iterates through its children in the same way as done by the other types of composite nodes, but it ignores their return value.
As can be seen in the example, each composite node needs to be named.
Properly naming things is considered one of the hardest things in computer science \cite{Fowler}, and it was found that the requirement to name composite nodes added significant developer overhead.

Also, as Python lacks the ability to do forward declarations of objects, there is a significant amount of jumping back and forth when configuring the tree.
Note for example that the \emph{CLEAN\_ROOM} task is referred to throughout the example.
This is necessary as its children need to be created before they can be assigned to the parent node.
It was found that this requirement to essentially build the tree from the bottom up was harder to do than to build the tree from the top down.

\subsection{ROS-Behavior-Tree}

\begin{listing}
    \inputminted[fontsize=\scriptsize]{Python}{\rootfolder/Chapters/Chapter3/Listings/rosbehaviortree.cpp}
    \caption[Implementation of a behavior tree using ROS-Behavior-Tree]{Implementation of a behavior tree using ROS-Behavior-Tree, from the ROS-Behavior-Tree documentation.}
    \label{lst:rosbehaviortree}
\end{listing}

ROS-Behavior-Tree is implemented in \CC, and like Pi Trees it is created specifically for use with \acrshort{ros}.
The tree is implemented in code, as shown in \cref{lst:rosbehaviortree}.
As the tree is implemented in code the same added complexity is seen here as described above.
Furthermore, as \CC is a compiled language, any change to the tree requires that the project is rebuild.

\subsection{behavior3}

behavior3 is implemented as both a Python library, \emph{behavior3py}, as well as a Javascript library, \emph{behavior3js}.
Additionally, the project includes an editor, \emph{behavior3editor}.
Of the options evaluated, behavior3 was the only one that allows for specifying the tree in a text file.
An attempt has been made to implement a tree using one of the other libraries, but it was quickly discovered that specifying the tree in code, either \CC or Python, became unwieldy as the tree size increased.

After evaluating the described options, it was decided to use behavior3.
The editor provides a graphical interface for creating behavior trees, and allows the user to export these as JSON configuration files.
The configuration files can be parsed and run using behavior3py or behavior3js.
As \acrshort{ros} allows for nodes to be implemented in \CC or Python, this project will use behavior3py.

%----------------------------------------------------------------------------------------
%	SECTION: IMPLEMENTATION
%----------------------------------------------------------------------------------------

\section{Implementation}

As described in \cref{ch:background}, a behavior tree is build from standard composite nodes, as well as nodes that implement decorators and actions.
To use behavior3py a \acrshort{ros} node was made, called \emph{cyborg\_bt}, as well as a node for organizing custom decorators and actions called \emph{cyborg\_bt\_nodes}.
cyborg\_bt imports the required decorators and actions needed by the Cyborg control system from cyborg\_bt\_nodes and runs the tree produced with behavior3editor.
Much of the functionality required to implement these decorators and actions has been placed in \acrshort{ros} services or topics, in order to make behavior as reusable as possible.

An example of such a node, the action node MoveTo, which moves the Cyborg to a requested location, is included with the attached source code.
The node issues a movement request to the navigation subsystem, and then monitors the progress of the Cyborg as it moves along the path to its target.
If progress stalls for longer than an allowed timeout, the movement request is aborted.
In this case, we monitor the movement along the path to the target and not merely the Euclidean distance to the target.
This is important, as the Cyborg may have to navigate around obstacles along its path which can involve moving further away from the target.

\begin{wrapfigure}{R}{0.5\columnwidth}
    \pimage[0.48]{Figures/patrol}
    \caption{Behavior tree that checks battery state, and charges if necessary, while patrolling.}
    \label{fig:patrol-tree}
\end{wrapfigure}

A simple demonstration of a working behavior tree is shown in \cref{fig:patrol-tree}.
This behavior tree causes the Cyborg to move in a patrol pattern between several locations while it monitors the state of the Cyborg's battery charge, and recharges as needed.

\end{document}

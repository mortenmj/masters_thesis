% Chapter Template
\providecommand{\rootfolder}{../..} % Relative path to main.tex
\documentclass[\rootfolder/main.tex]{subfiles}
\begin{document}

\chapter{Background}
\label{ch:background}

The Cyborg is composed of a number of major and minor hardware and software components.
In the following sections, these will be described in order to provide the necessary background for the subsequent steps undertaken in this thesis to implement the previously described functionalities.

%----------------------------------------------------------------------------------------
%	SECTION: HARDWARE
%----------------------------------------------------------------------------------------

\section{Hardware}

The Cyborg itself is a MobileRobots Pioneer LX base, with additions made by the Cyborg project.
In addition to a custom power supply, the Cyborg has a Stereolabs \emph{ZED} stereo camera, as well as a Nvidia \emph{Jetson TX2} embedded computer for image processing.
Here, some detail will be given on each hardware component that is relevant to this thesis.

%----------------------------------------------------------------------------------------
%	SUBSECTION: PIONEER LX
%----------------------------------------------------------------------------------------

\subsection{Pioneer LX}

\begin{figure}[h]
    \centering
    \subcaptionbox{MobileRobots Pioneer LX.\label{fig:robot-original}}[0.48\columnwidth]{\pimage[0.48]{Figures/robot}}%
    \subcaptionbox{NTNU Cyborg.\label{fig:robot-ntnu}}[0.48\columnwidth]{\pimage[0.48]{Figures/final-transp}}
    \caption{Comparison of the original Pioneer LX and the Cyborg.}
    \label{fig:robot-comparison}
\end{figure}

The NTNU Cyborg uses the Pioneer LX robot, from MobileRobots~(\cref{fig:robot-original})~\cite{MobileRobots}.
Omron Adept MobileRobots, previously MobileRobots, is a manufacturer of intelligent mobile robots for commercial and industrial use.

The Pioneer LX comes equipped with the required sensors for safe, autonomous operation.
The robot has a \emph{SICK S300} \ang{270} laser rangefinder for navigation and object detection, in addition to a \emph{SICK TiM 510} laser for frontal sensing near floor level.
Additionally, the robot has rear facing ultrasonic sonar sensors and front bumpers for collision detection.
These sensors allow for \acrfull{slam}, which enables the robot to navigate without pre-mapping the environment.
It is designed for continuous operation for 13 hours before recharging, which can be performed autonomously.

%----------------------------------------------------------------------------------------
%	SUBSECTION: STEREOLABS ZED CAMERA
%----------------------------------------------------------------------------------------

\subsection{Stereolabs ZED camera}

\begin{figure}[h]
    \pimage{Figures/zed}
    \caption{Stereolabs ZED camera.}
    \label{fig:zed-camera}
\end{figure}

The Stereolabs ZED~(\cref{fig:zed-camera}) is a dual 4 megapixel camera.
It provides 1920$\times$1080 video at 30 \acrfull{fps} frames per second, or 800$\times$400 at 100 \acrshort{fps}, with a \ang{110} angle of view~\cite{Stereolabs}.

The camera is designed to emulate the stereoscopic operation of human eyes.
By recording an image from two slightly offset points, depth and motion in space can be inferred by comparing the displacement of pixels in the left and right images.
This allows for the camera to provide a depth map of the recorded scene, where pixels are defined by an (x, y, z)-tuple rather than the normal (x, y) coordinate pair.
The information is exported as a black and white depth map.
By identifying the pixels of an object, the distance to the object can be determined by calculating the mean intensity of these pixels in the depth map~\cite{Stereolabs}.

%----------------------------------------------------------------------------------------
%	SUBSECTION: NVIDIA JETSON TX2
%----------------------------------------------------------------------------------------

\subsection{Nvidia Jetson TX2}

\begin{wrapfigure}{R}{0.5\columnwidth}
    \pimage[0.49]{Figures/tx2}
    \caption[Nvidia Jetson TX2 development kit.]{Nvidia Jetson TX2 development kit, image courtesy of Nvidia.}
    \label{fig:tx2}
\end{wrapfigure}

The Nvidia Jetson TX2, shown in~\cref{fig:tx2}, is an embedded computer from Nvidia, designed for real-time data processing for \emph{artificial intelligence}\footnote{https://www.nvidia.com/en-us/autonomous-machines/embedded-systems-dev-kits-modules/}
The stated purpose of the device is to move data processing from a central location to the edge, meaning on the robot or drone itself.
The TX2 has a small form factor of 17 by 17 centimeters, and a power consumption of 7.5 watts under normal load.
Despite the low footprint in size and power consumption, the onboard GPU delivers in excess of 1 T\acrshort{flops} ($10^{12}$ \acrlong{flops}) of computing power~\cite{Nvidia2017}.

%----------------------------------------------------------------------------------------
%	SECTION: SOFTWARE
%----------------------------------------------------------------------------------------

\section{Software}

The Cyborg's onboard computer comes pre-installed with Ubuntu Linux and \acrshort{ros}.
MobileRobots provide a number of software packages for efficient use of the robot. 
Among these are the \acrfull{aria} library, the \acrfull{arnl} library and the software applications MobileSim and MobileEyes, which enable implementation of advanced functionality in ROS.

%----------------------------------------------------------------------------------------
%	SUBSECTION: ROS
%----------------------------------------------------------------------------------------

\subsection{\acrfull{ros}}

The \acrfull{ros}\footnote{https://www.ros.org} is an open-source framework for writing robot control software.
It consists of a collection of libraries and tools that simplify the task of implementing complex and robust control software for different robot platforms.
It is not an operating system in the traditional sense of managing and scheduling processes but acts as middleware, providing a communications layer for a distributed control system.

The framework provides middleware for individual software modules, known as nodes, so that modules from various sources can interact using a standard interface.
These can all be implemented separately and can communicate with each other using the \acrshort{ros} communications layer.
This approach simplifies development, by allowing components developed by separate individuals, teams or organizations to communicate using a standardized interface.
This promotes collaboration and code reuse, enabling different research teams to focus on a subset of robot control rather than building a complete solution from the ground up.

The following gives a run-down of the core concepts in \acrshort{ros}.
The components of \acrshort{ros} are \emph{nodes}, \emph{messages}, \emph{topics} and \emph{services}~\cite{Quigley2009}.

\subsubsection{Nodes}

Nodes are processes that perform computation, and are primarily implemented in Python or C++.
Nodes are self-contained processes, and communicate with other nodes by passing messages through the \acrshort{ros} middleware.
\acrshort{ros} modules are generally constructed from a number of nodes.
This approach follows the Unix philosophy of \say{do one thing and do it well}, which simplifies software development by limiting the scope of each individual node.
\acrshort{ros} allows for a control system to be divided into any number of nodes, which may run on the same machine or on different machines~\cite{ROS}.

When implementing nodes in C++, it is also possible to implement them as nodelets.
A nodelet is implemented similarly to a normal node, but is instantiated by a special node, the \emph{nodelet manager}.
This can be done by creating a separate node, which loads all the required nodelets, or it can be done using a launch file.
The advantage of running multiple nodelets in one node is that communication is done using shared memory, thereby avoiding the overhead of \acrshort{tcp} communication~\cite{Quigley2009}.

\subsubsection{Messages}

The information passed between nodes is encapsulated in messages, which may be thought of as analogous to structs in C.
Messages are composed of standard data types such as integers, floats and so on, and arrays of these.
A message can also be composed of other messages.
Many standard message formats are available in \acrshort{ros}, and it is also possible to define custom message types.
Nodes publish messages in one of two ways, either by broadcasting them to a topic or by answering service requests~\cite{Quigley2009}.

\subsubsection{Topics}

Messages can be published to a topic, that can be subscribed to by other nodes.
This publish-subscribe model allows for sharing of data in a broadcast manner, and there may be any number of publishers and subscribers for a topic.
One example of such a node is the feedback node, from move\_base, which publishes the current position of the robot.
Any other node that wants to know the current position can subscribe to this, and will receive the new position whenever the robot moves~\cite{Quigley2009}.

\subsubsection{Services}

While the publish/subscribe model works well for many types of information, it is necessarily an asynchronous form of communication.
For applications where synchronous communication is required, it is possible to use services.
A service is defined by two messages, the input message and the result message.
When called, the service will execute and the result is returned to the caller.
Services can be used to request information, request the execution of a physical action, or some other task.
\Cref{lst:ros_service} shows one such service, in use by the Cyborg project~\cite{Quigley2009}.

\begin{listing}
    \inputminted{python}{\rootfolder/Chapters/Chapter2/Listings/distance_to_goal.py}
    \caption{Example of an \acrshort{ros} service, written by the author.}
    \label{lst:ros_service}
\end{listing}

\subsubsection{Actions}

While service calls are useful for remote procedure calls which execute quickly, they are blocking and should be avoided for long-running tasks, or tasks that may have to be preempted~\cite{ROS}.
For this purpose, \acrshort{ros} provides actions.
Actions are used for procedure calls that cause the robot to perform a long-running task, such as moving to a location or some other real-world action.
Actions are able to keep state for the lifetime of a provided goal, and will provide feedback to each client that issues a goal to the server~\cite{Quigley2009}.

\subsection{ARIA}

\emph{\acrfull{aria}} is a \CC library for all robots from MobileRobots.
The library allows for dynamic control of the robot's velocity, heading, relative heading as well as other parameters.
This can be achieved both using a low-level interface and a higher-level Actions infrastructure.
\acrshort{aria} receives sensor data from the robot platform, such as position estimates, sonar data and laser rangefinder data.
While written in C++, the \acrshort{aria} library can be accessed from other languages, amongst others Python~\cite{ARIA}.

\subsection{ARNL}

\emph{\acrfull{arnl}} is \CC library from MobileRobots built on top of \acrshort{aria} which provides intelligent navigation and localization.
\acrshort{arnl} provides information about the current position of the robot, and an interface for requesting that the robot move to a given location.
The software updates the current position automatically, using data from the robot's sensors and map.

These features are provided in \acrshort{ros} as a node, called \emph{ros-arnl}, which exposes this functionality in the form of topics and services.
This node provides a simple interface for higher-level software to monitor and control the position of the robot~\cite{ARNL}.

%----------------------------------------------------------------------------------------
%	SUBSECTION: MOBILEEYES
%----------------------------------------------------------------------------------------

\subsection{MobileEyes}

\emph{MobileEyes} is a graphical interface from MobileRobots for monitoring robot motion and sensor output.
The program allows the user to monitor the movement of the robot on the map, and it is possible to send commands to the robot remotely.
The software also allows for reconfiguring the robot, and it is possible to send custom commands or create custom overlays that are shown on the map~\cite{MobileEyes}.

%----------------------------------------------------------------------------------------
%	SUBSECTION: MOBILESIM
%----------------------------------------------------------------------------------------

\subsection{MobileSim}

\emph{MobileSim} is a simulation software package from MobileRobots, which allows for testing \acrshort{ros} modules in simulation.
The software emulates the physical robot so that other parts of the \acrshort{ros} integrate without any necessary changes.
This simulation includes data streams from sensors such as the sonar and laser rangefinders, which allows for efficient testing of control software during development~\cite{MobileSim}.

%----------------------------------------------------------------------------------------
%	SUBSECTION: ROS 2
%----------------------------------------------------------------------------------------

\subsection{ROS 2}

Here, we will give an overview of \emph{\acrshort{ros} 2}\footnote{http://www.ros2.org/}, and how the project has attempted to remedy some perceived shortcomings in \acrshort{ros}.
As outlined in~\cite{Gerkey2017}, \acrshort{ros} was developed based on the use case of the Willow Garage \emph{PR2} robot.
Due to this, development was guided by the characteristics of the PR2 robot, including

\begin{itemize}
    \item Support for a single robot.
    \item Significant computational resources available.
    \item No real-time requirements.
    \item Strong network connectivity available.
    \item Mainly academic applications.
\end{itemize}

Since the beginning of \acrshort{ros} in 2007, several of these assumptions have changed, and \acrshort{ros} has been used on a far wider range of robots than for what it was originally designed~\cite{Gerkey2017}.
Among the new use cases outlined, are

\begin{itemize}
    \litem{Control of multiple robots}
        Currently, there is no standard way to control more than a single robot using \acrshort{ros}.
        \acrshort{ros} has a single-master architecture, and multi-robot support does not elegantly integrate into this design.
    \litem{Limited computational resources}
        \acrshort{ros} is not designed to run on micro controllers. Therefore, nodes must interact with these through a device driver.
        \acrshort{ros} 2 is designed so that these controllers can be implemented as nodes, and thereby participate directly in the control system as first-class citizens.
    \item{Built-in real-time support.}
    \litem{Non-ideal networks}
        As seen in earlier work on this project by~\cite{Waløen2017}, \acrshort{ros} does not degrade gracefully when run on unreliable networks.
        \acrshort{ros} 2 aims to alleviate this.
    \litem{Academic and industrial applications:}
        \acrshort{ros} 2 aims for \acrshort{ros} to remain the platform of choice in academic robotics, while also becoming increasingly relevant in industrial applications.
\end{itemize}

The central feature of \acrshort{ros} is the publish-subscribe middleware which allows for loose coupling of individual nodes.
As \acrshort{ros} was begun in 2007, there was no sufficiently mature off-the-shelf technology that provided this, and this system was built essentially from scratch.
The \acrshort{ros} project implemented the necessary framework for node discovery, message definition, serialization and transport.
Since 2007, a number of technologies have matured that provide this capability, and it would not have been necessary to build a custom solution today.
Several advantages to using one of these off-the-shelf technologies are listed by~\cite{Gerkey2017}:

\begin{itemize}
    \item Less code to be maintained by the project developers
    \item Third party solutions may offer features outside the scope of what the project could develop themselves
    \item The project benefits from ongoing improvements to third party solutions
    \item Third party solutions may be rigorously proven, and thereby improve the perceived reliability of \acrshort{ros}
\end{itemize}

%----------------------------------------------------------------------------------------
%	SECTION: BIOLOGICAL NEURAL NETWORKS
%----------------------------------------------------------------------------------------

\section{Biological neural networks}

\begin{figure}
    \pimage{Figures/biological-neuron}
    \caption[Model of a biological neuron.]%
            {Model of a biological neuron, courtesy of Blausen Medical.}
    \label{fig:biological_neuron}
\end{figure}

As a long-term goal, the Cyborg project hopes to use biological neurons in a robot control loop.
Here, we will give some background on relevant biological concepts.
Further detail on these concepts can be found in~\cite{Knudsen2016}.

\emph{Neurons} are electrically excitable cells, which process and transmit information using electrical and chemical signals.
The signals travel via synapses, which are specialized connections between neurons.
The sum of neurons and their connections are referred to as a \emph{neural network}.
They are the core components of the central nervous system, and along with the \emph{ganglia} form the core of the of the peripheral nervous system.

\begin{wrapfigure}{R}{0.5\columnwidth}
    \pimage[0.49]{Figures/brain-hierarchy}
    \caption[Model of the hierarchical structure of the brain.]%
            {Model of the hierarchical structure of the brain~\cite{Perry1999}.}
    \label{fig:brain-hierarchy}
\end{wrapfigure}

When the \emph{membrane potential} of a neuron is excited past a certain threshold, an \emph{action potential} is triggered.
An action potential is a rapid rise and fall in the membrane potential of the neuron, which causes the \emph{axon hillock} to fire an electrical signal down the \emph{axon} of the neuron.
The membrane potential of the neuron is excited by the firing of upstream neurons, and other extracellular \emph{potential changes}.

A \emph{synapse} is a structure which connects two neurons, and allows for the transmission of an electrical or chemical signal.
Synaptic communication generally travels along axons and synaptic \emph{terminals} of the upstream neuron to the \emph{dendrites} of the downstream neuron.
Synapses may either transmit electric signals directly, in \emph{electrical synapses}, or by using neurotransmitters, in \emph{chemical synapses}.

The neurons in the brain are organized in a hierarchical manner.
Signals enter through the \emph{brain stem}, and travel upwards as shown in~\cref{fig:brain-hierarchy}.
As shown, higher layers of the brain are responsible for increasingly sophisticated levels of thought.
This layering is also present in the \emph{neocortex}.
Taking vision as an example, lower levels of the neocortex are responsible for simple features such as edges and corners.
Low-level patterns are combined at mid-levels into more complex features such as curves and textures.
Finally, at higher levels of the neocortex, complex objects such as cars and houses are recognized.

%----------------------------------------------------------------------------------------
%	SECTION: ARTIFICIAL NEURAL NETWORKS
%----------------------------------------------------------------------------------------

\section{Artificial neural networks}

\begin{wrapfigure}{R}{0.4\columnwidth}
    \iimage[0.39]{Figures/artificial-neuron}
    \caption{Model of an artificial neuron.}
    \label{fig:artificial_neuron}
\end{wrapfigure}

The Cyborg is intended to move around in its surroundings, and is equipped with a stereo camera in order to orient itself.
Here, an overview is given of the techniques used in computer and robot vision, and the research that has driven the explosion of activity within this field in the past few years.
This section provides some background on how artificial neural networks relate to robot vision, as well as an overview of how they function in general.

\emph{Artificial neural networks} are an approach to perform complex computational tasks as an emergent process of a large number of simple interconnected units.
This approach is inspired by the activity of neurons in the brain.
The artificial neuron is modelled by an \emph{activation function}, which outputs a value as a function of the sum of its inputs.
Historically, this has been one of several possible \emph{sigmoid functions}, for example $g\left(x\right) = \frac{1}{1 - e^{-x}}$.
An illustration of an artificial neuron is shown in~\cref{fig:artificial_neuron}.

\emph{Nodes} in artificial neural networks are organized into \emph{layers} of units which are connected to the units in the consequent and previous layers.
The value of the signal flowing into a node is a function of the value flowing out of the previous layer, and the weights assigned to the particular connections.
A neural network with a sufficient number of units and a continuous, bounded and non-constant activation function, is able to approximate any mathematical function~\cite{Cybenko1989}\cite{Hornik1991}.

\begin{wrapfigure}{R}{0.4\columnwidth}
    \iimage[0.39]{Figures/xor-net}
    \caption{\acrshort{xor} network, illustrating how neurons can implement basic logic functions.}
    \label{fig:xor_net}
\end{wrapfigure}

A simple example, which approximates the \acrfull{xor} function, is illustrated in~\cref{fig:xor_net}.
The weights in this example are determined by construction.
In a real scenario, the weights are found by minimizing some \emph{cost function} through the process of \emph{gradient descent}, which is referred to as training the network~\cite{Mitchell1997}.
Through training, the network would find a different set of weights while still achieving the same output.

%----------------------------------------------------------------------------------------
%	SUBSECTION: CONVOLUTIONAL NEURAL NETWORK (CNN)
%----------------------------------------------------------------------------------------

\subsection{\acrfullpl{cnn}}

\begin{figure}
    \iimage{Figures/mnist-net}
    \caption[An example of a neural network with one fully connected hidden layer.]%
            {An example of a neural network with one fully connected hidden layer. %
            The input to the network is a 28$\times$28 pixel image of a single digit, flattened to a 784 element vector. %
            The output is a confidence score for each of the possible digits. %
            The network was trained by the author, and achieved 92.4\% verification accuracy on the \acrshort{mnist} handwritten digit dataset. %
            More complex networks exceed 99\% accuracy on this dataset~\cite{mnist2010}.}
    \label{fig:mnist-net}
\end{figure}

\begin{figure}
    \pimage{Figures/features}
    \caption[Structure of a \acrfull{cnn}.]{Structure of a \acrfull{cnn}~\cite{Mathworks}.}
    \label{fig:cnn-classification}
\end{figure}

For a fully connected network as the one shown in~\cref{fig:mnist-net}, every neuron in the \emph{hidden layer} is connected to each input.
In this case there are 784 connections, one for each pixel in the input image.
However, with one input per color channel, per pixel, this number would quickly balloon into the hundreds of thousands for a larger image.
Instead of flattening an image to a column vector, as done in the fully connected network shown before, a \acrshort{cnn} arranges its neurons in a 3D volume corresponding to the width, height and color channel depth of the input image.
Furthermore, each neuron in a hidden layer is connected only to a smaller region of the previous layer, as shown in~\cref{fig:cnn-classification}.
Between \emph{convolutional layers}, \emph{pooling layers} have traditionally been inserted to reduce the spatial size of the network.
This decreases the computational complexity of the network, and by lowering the number of parameters also limits \emph{overfitting}~\cite{Szegedy2014}.

\begin{figure}
    \pimage[0.9]{Figures/abstract-features}
    \caption[Neural network learning increasingly abstract features.]%
            {Neural network learning increasingly abstract features~\cite{Brown2015}.}
    \label{fig:abstract-features}
\end{figure}

A \acrshort{cnn} consists of convolutional, pooling, and fully connected layers, which perform the following tasks.
A convolutional layer consists of a set of learned \emph{filters}.
These filters are moved across the input image, and as the network is trained it learns filters that activate when presented with some visual feature.
As shown in~\cref{fig:abstract-features}, early layers may learn basic features such as horizontal or vertical lines.
Later layers learn increasingly high-level features, such as the shape of an eye, the corner of a mouth or some other building block of an image~\cite{Brown2015}.

%----------------------------------------------------------------------------------------
%	SUBSECTION: IMAGE CLASSIFICATION
%----------------------------------------------------------------------------------------

\subsection{Image classification}

\begin{figure}
    \iimage[0.9]{Figures/ilsvrc}
    \caption[The influence of deep learning on image classification error rates.]%
            {Progress of image classification, and the growth of deep networks. %
            The graph shows the top-5 classification errors of each year's \acrshort{ilsvrc} winner, and the depth of the network used. %
            The 2011 winner did not employ a neural network solution.
            The data for this figure is from~\cite{Krizhevsky2012},~\cite{Zeiler2013},~\cite{Szegedy2014} and~\cite{He2016}.}
    \label{fig:ilsvrc}
\end{figure}

\begin{wrapfigure}{R}{0.5\columnwidth}
    \pimage[0.49]{Figures/activation}
    \caption{Illustration of Sigmoid and \acrshort{relu} activation functions.}
    \label{fig:activation-functions}
\end{wrapfigure}

\emph{Image classification} refers to the task of assigning images to one of a set of possible classes, based on the contents of the image.
Several competitions are hosted yearly, where research teams compare progress in detecting a wide variety of objects.
While neural networks have been known since they were first described in literature in 1943~\cite{Mitchell1997}\cite{Mcculloch1943}, their utility in complex tasks such as this has only begun to be developed in the past few years.
This development has largely been attributed to advances in available computing power, the size of available data sets and algorithmic advances which allow for training larger neural networks.

As explained in~\cite{Krizhevsky2012}, the ability to correctly classify complex images into one of several thousand possible categories requires a model with a large learning capacity.
However, even with the millions of images in a dataset such as ImageNet it is infeasible to train an fully connected neural network to perform this task.
\acrlongpl{cnn} enable networks to detect features in images, with a lower computational complexity than standard nets.
At the most basic levels of the network, such features may be simple horizontal and vertical lines while later layers may learn more abstract features.
This is illustrated in~\cref{fig:abstract-features} and~\cref{fig:cnn-classification}.
Compared to fully connected networks, such as the one illustrated in~\cref{fig:mnist-net}, \acrshortpl{cnn} have fewer connections and parameters and are therefore easier to train.

Until 2012, the top performing algorithms in the yearly \acrfull{ilsvrc} was dominated by algorithms requiring a large amount of manual hand coding of features, and that still had an error rate of over 26\%.
In 2012, researchers from the University of Toronto presented a deep \acrshort{cnn}, made possible by using a new activation function termed the \emph{\acrfull{relu}}, see~\cref{fig:activation-functions}.
While the Sigmoid function quickly saturates which slows down training in deeper layers, the \acrshort{relu} function does not.
This makes networks using \acrshort{relu} less susceptible to disappearing gradients, which has allowed for training of a deeper network than previously possible~\cite{Krizhevsky2012}.
Their network was entered in the 2012 \acrshort{ilsvrc} competition, and was able to classify images with an error rate of 15.3\%.
This outcompeted previous winners by over 10 percentage points, as shown in~\cref{fig:ilsvrc}.
While the concept of \acrshortpl{cnn} and deep learning had been known for many years, this was the first significant use of such networks in computer vision.

\begin{figure}
    \pimage[0.49]{Figures/resnet}
    \caption[Illustration of a residual block.]{Illustration of a residual block~\cite{He2016}.}
    \label{fig:residual-block}
\end{figure}

Later advances have improved the applicability of deep neural networks further.
In 2015, the authors of~\cite{He2016} proposed a novel technique of feed-forward from the input signal, leading to great improvement of the trainability of extremely deep networks.
As explained, a neural network can approximate any mathematical function.
The key insight of the authors is that if a network of a given depth can approximate such a function, and a functionally identical deeper network can be created by inserting unity layers, then a deeper network should not yield poorer performance than the original network.
However, the result found in practice was that as network depth increases, network performance flattens out and eventually degrades.
The authors proposed that this is a problem of training the network, rather than a fundamental problem of extremely deep networks.
To overcome this, the authors note that the desired mapping is likely to be closer to its input $\vec{x}$, than a zero mapping.
If the original desired mapping is the function $\mathcal{H}(\vec{x})$, the network is instead trained to approximate $\mathcal{F}(\vec{x}) = \mathcal{H}(\vec{x}) - \vec{x}$.
This greatly improves the trainability of the network and allows for the increase in network depth shown in~\cref{fig:ilsvrc}.
Unlike previous approaches, the network achieves high accuracy through a deep but simple and repeating architecture composed of residual blocks as shown in~\cref{fig:residual-block}.
This approach gives the network a computational complexity much lower than the number of layers might seem to imply~\cite{He2016}.

As deeper networks are able to capture more abstract features in images, machine vision becomes more robust and consequently applicable to a larger range of tasks.
The organizers of \acrshort{ilsvrc} have announced that the 2018 competition will involve classifying 3D objects as well as 2D images.

%----------------------------------------------------------------------------------------
%	SUBSECTION: OBJECT DETECTION
%----------------------------------------------------------------------------------------

\subsection{Object Detection}

In image classification tasks, the algorithm is trained to classify an image of a single object, or alternatively the principal object in the image for images with more than one object.
In \emph{object detection} tasks, the goal is to locate and classify all objects in the image, including identifying the boundary between objects and how they relate to one another~\cite{Girshick2013}.
Just as deep learning has revolutionized object classification, great strides have been made in object detection in recent years.
As these developments are highly relevant to the field of robotics and the Cyborg project, this section gives an overview of these developments.

The same networks that are used for classifying images with a single object can also be applied to classify individual objects within a more complex image.
The challenge in object detection is to identify these individual objects, so that they can be classified in the manner described in the previous section~\cite{Girshick2013}.

Similar to the progress seen in object classification, previous to the deep learning revolution the best-effort approaches used other techniques than neural networks~\cite{Girshick2013}.
The very earliest attempts at object detection using \acrshortpl{cnn} employ a naive technique of moving a \emph{sliding window} across the image.
Using windows of various sizes and classifying these sub-images one by one, it is possible to detect objects within an image composed of multiple objects.
However, this is a brute force approach and for fine-grained detection requires classifying thousands of windows per image.

\subsubsection{\acrfull{rcnn}}

\begin{figure}
    \pimage{Figures/rcnn}
    \caption[Object detection and classification using \acrshort{rcnn}.]{Object detection and classification using \acrshort{rcnn}~\cite{Girshick2013}.}
    \label{fig:r-cnn}
\end{figure}

The approach taken by~\cite{Girshick2013}, called \emph{\acrfull{rcnn}} extracts region proposals from the image by a process called \emph{selective search}.
The selective search algorithm looks at the image through windows of different sizes, and identifies relevant regions by grouping adjacent pixels by color, texture or intensity.
The resulting groups are reshaped into bounding boxes, and the contents of each bounding box are fed to an image classifier.
If the contents of a box are successfully classified, the algorithm attempts to tighten the bounding box using a linear regression model.
The process is shown in~\cref{fig:r-cnn}.
The approach has low error rate, but the connection of three different models leads to high complexity which makes the system difficult to train.

\subsubsection{Fast \acrshort{rcnn}}

While the approach of generating region proposals is significantly less computationally expensive than using sliding windows, it still requires that around 2000 region proposals are classified.
In a follow-up paper, Gerschick proposes an approach that is improved in two significant ways called \emph{Fast \arcshort{rcnn}}.
Firstly, rather than running individual regions of the image through the feature classifier one by one, features are computed on the entire image in a single pass to create a feature map, in a process called Region of Interest Pooling.
Subsequently, the features for each region can be obtained by selecting the appropriate area from the pre-computed feature map.
Secondly, the three models used previously (region proposer, image classifier and bounding box regression model), are combined into a single model.
This allows for end-to-end training, which greatly improves trainability~\cite{Girshick2015}.

\subsubsection{Faster \acrshort{rcnn}}

\begin{wrapfigure}{r}{0.4\columnwidth}
    \pimage[0.39]{Figures/faster-rcnn}
    \caption[Object detection using Faster \acrshort{rcnn}.]{Object detection using \acrshort{rcnn}~\cite{Ren2017}.}
    \label{fig:faster-rcnn}
\end{wrapfigure}

While the advances outlined above greatly improve on the efficiency of object detection algorithms, they expose the region proposer as a significant bottleneck~\cite{Ren2017}.
Selective search is used to generate region proposals, while a \acrshort{cnn} is used to extract features, classify the image and compute a bounding box.
The work done by~\cite{Ren2017} uses the features computed by the \acrshort{cnn} discussed previously, and combines this with a separate \acrshort{cnn}, called the \emph{Region Proposal Network}.
They name this approach \emph{Faster \arcshort{rcnn}}.
By making use of the same feature map that is used to classify images, the authors enable essentially cost-free region proposals.
The network passes a sliding window over the feature map, and computes region proposals along with an objectness-score.
The objectness-score measures the probability that the region contains an object, and allows for selecting only the regions that meet some minimum threshold.

\subsubsection{Mask \acrshort{rcnn}}

\begin{figure}[H]
    \pimage{Figures/mask-rcnn}
    \caption[Pixel-level object detection using Mask \acrshort{rcnn}.]{Pixel-level object detection using Mask \acrshort{rcnn}~\cite{He2017}.}
    \label{fig:mask-rcnn}
\end{figure}

Later work extends this approach by detecting which pixels belong to each of the detected objects, as shown in~\cref{fig:mask-rcnn}~\cite{He2017}, a method named \emph{Mask \acrshort{rcnn}}.
Here, in parallel to the region proposal network, a network is branched off which simultaneously computes a binary mask for each object.
The binary map identifies which pixels belong to the object, as shown in~\cref{fig:mask-rcnn}.

\subsubsection{\acrfull{yolo}}

\begin{figure}[H]
    \pimage{Figures/yolo}
    \caption[Object detection using \acrshort{yolo}.]{Object detection using \acrshort{yolo}~\cite{JosephRedmon}.}
    \label{fig:yolo}
\end{figure}

The network used by this project, named \acrfull{yolo}, takes a novel approach.
Rather than making a first pass to detect shapes, and a second pass to classify them, these operations are performed by a single neural network.
The image is divided into a grid, and the probability that each cell in the grid forms part of a bounding box, as well as the probability that the contents of the box belongs to an image class, is computed simultaneously.
These values are then combined to obtain a class-specific confidence score which encodes both the probability of the particular class appearing in the box, and how well the box fits the object.
This allows the network to reason globally about the contents of the image, where the \acrshort{rcnn} approach treats each selected shape separately from the others.
Furthermore, performing detection in a single pass provides greater speed~\cite{Redmon2015}.

%----------------------------------------------------------------------------------------
%	SUBSECTION: TRANSFER LEARNING
%----------------------------------------------------------------------------------------

\subsection{Transfer learning}

\begin{figure}
    \pimage{Figures/nvidia}
    \caption[Free-space detection and 3D object detection for autonomous driving.]{Free-space detection and 3D object detection for autonomous driving~\cite{NVIDIA}.}
    \label{fig:nvidia-cnn}
\end{figure}

The level of performance discussed in the previous section requires the training of very large convolutional neural networks.
Due to problems of overfitting in such large networks, it is necessary to train the network on very large datasets.
In~\cite{NVIDIA}, researchers from NVIDIA describe a process by training such a network on the ImageNet dataset, consisting of approximately 1.2 million images.
These are in turn augmented by various transformations to a total dataset of 22 million images.
By training on such a large dataset, it is possible to create a very robust feature detector.
However, the amount of training required can take weeks or months, as in the example described by the researchers.

The features learned by a convolutional neural network, as shown in~\cref{fig:abstract-features}, have been found to be similar across many different applications.
Observe in~\cref{fig:cnn-classification}, that the majority of a network performs feature detection while only the last few layers use these features to classify the image.
One application of transfer learning is adapting an existing network to a new domain.
This can be done by replacing erasing the weights of the classification part of the network, or replacing these layers with new layers if required.
By keeping the weights of the feature learning layers of the network fixed, the network can be retrained to a new purpose.
In the case of the work done at NVIDIA, a general image classifier trained on the ImageNet dataset was repurposed to identify objects in the autonomous vehicle domain.
This can then be done much faster, and with a much smaller dataset, than what is required to train a new network.
Using this technique resulted in a network able to detect pedestrians and vehicles, and identify which sections of the road are safe for driving, as shown in~\cref{fig:nvidia-cnn}\cite{NVIDIA}.

%----------------------------------------------------------------------------------------
%	SECTION: FINITE STATE MACHINES
%----------------------------------------------------------------------------------------

\section{Finite State Machines}

At the outset of the work on this thesis, the cyborg was running the decision making software outlined in~\cite{Andersen2017}.

While state machines are a well-proven approach to flow control in many applications, there was a desire to investigate alternative approaches.
This and the following section will provide some background on these two approaches.
These concepts are explained in detail in~\cite{Millington2009}.

\begin{figure}
    \iimage[0.39]{Figures/statemachine}
    \caption{An example of a simple state machine.}
    \label{fig:fsm}
\end{figure}

A \emph{finite state machine} is an abstract model of computation, that allows for actions to be encoded into a finite number of states.
Changes from one state to another are referred to as \emph{transitions}.
The transition from an initial state to a new state is governed only by the starting \emph{state}, and the \emph{event}, as shown in~\cref{fig:fsm}.
Whenever the conditions for a transition are met, the system will perform that transition and execute the \emph{action} associated with the new state.
State machines typically have actions associated with entering and exiting a state.

%----------------------------------------------------------------------------------------
%	SUBSECTION: EVENTS
%----------------------------------------------------------------------------------------

\subsection{Events}

The type of state machines described here is termed an event-driven state machine.
As the \acrshort{ros} system needs to perform many different actions, it is necessary that we do not block the system by polling for new events.
Instead, events arrive asynchronously and are consumed by the state machine before it goes back to sleep.

%----------------------------------------------------------------------------------------
%	SUBSECTION: ACTIONS
%----------------------------------------------------------------------------------------

\subsection{Actions}

Actions in finite state machines are associated with transitions.
In actual use, actions are typically modularized by splitting into an entry action and an exit action for each state as well as component executed repeatedly while the state is active.
Often, it is desirable to specify behavior that should trigger when the state machine enters or leaves a state.
This granularity achieved by splitting each actions simplifies behavior reuse, for example if every transition into a states share behavior that can be placed in an entry action.

%----------------------------------------------------------------------------------------
%	SUBSECTION: STRENGTHS AND WEAKNESSES
%----------------------------------------------------------------------------------------

\subsection{Strengths and weaknesses}
\label{sec:state_machine_strength_weaknesses}

An overview of the use of state machines for AI in computer games is given in.
In general, the concerns outlined here overlap with the concerns relevant in the sort of high-level decision making that is necessary in the Cyborg project.

State machines have work well for structured behavior, as they make it simple to implement a sequence of behaviors and the conditions for transitioning between them.
They work particularly well for behavior where it is necessary to interrupt the current behavior due to external events.

One major weakness of state machines is that as the control system grows, the number of states and transitions between states quickly balloons and can be come unmanageable.
Take as an example a control system for the Cyborg, where it should act in different ways depending on the charge level of its battery.
If the battery is in good condition, the Cyborg should behave as normal.
If the battery is in a medium condition, the Cyborg should attempt to minimize its energy use and should recharge as soon as is convenient.
And, if the battery is in poor condition, the Cyborg should do whatever it can to recharge as soon as possible.
Using a state machine, there are two ways to implement this.
Either, would be possible to use a hierarchy of state machines, one for each battery state.
Alternatively, each state in the state machine can be a state machine in itself, with a state for each battery state.
In both cases, the number of states would be three times the number of states in the original control system.
Furthermore, the number of transitions that must be added when including a new state grow with the size of the state machine, and the process can quickly become cumbersome.

Another weakness, which has been an important consideration for this project, is that state machines fall short when implementing unstructured behavior.
While they are well suited for structured behavior of the type that a robot may do in an industrial setting, they are less suited for a robot which is intended to roam freely and react in a life-like way to its surroundings~\cite{Millington2009}.

%----------------------------------------------------------------------------------------
%	SUBSECTION: IMPLEMENTATION IN ROS
%----------------------------------------------------------------------------------------

\subsection{Implementation in \acrshort{ros}}

The existing control system at the outset of this project was implemented using \acrshort{smach}, a task-level state machine architecture for \acrshort{ros}.
\acrshort{smach} allows for fast prototyping and implementing complex state machines.
Since it is a task-level architecture, it is not suitable for low-level control but rather for high-level decision making.
As mentioned in~\cref{sec:state_machine_strength_weaknesses}, state machines are not well suited to handle unstructured tasks.
This is also stated by the \acrshort{smach} developers~\cite{Bohren}.

States in \acrshort{smach} are implemented as individual classes, which allows for reuse.
This also includes reuse of behavior between states, by using object composition.
Actions to be performed by the robot are associated with states in the state machine, and the robot will carry out that action for as long as it is in the particular state.

\begin{listing}
\inputminted{python}{\rootfolder/Chapters/Chapter2/Listings/smachstate.py}
\caption{State example from the \acrshort{smach} documentation.}
\end{listing}

%----------------------------------------------------------------------------------------
%	SECTION: BEHAVIOR TREES
%----------------------------------------------------------------------------------------

\section{Behavior trees}

\begin{wrapfigure}{L}{0.5\columnwidth}
    \iimage[0.49]{Figures/behaviortree}
    \caption{An example of a simple behavior tree.}
    \label{fig:behaviortree}
\end{wrapfigure}

A behavior tree is a model for task execution commonly used for artificial intelligence in video games, but that is also applicable to robotics and other types of control systems.
Much of the strength of the behavior tree approach lies in the ability to compose complex behavior from simple building blocks, without needing to consider the implementation of each building block.
In this sense they are similar to finite state machines.
However, while the fundamental building block in a state machine is the state, the fundamental building block in behavior trees is the task.
These concepts are explained extensively in~\cite{Millington2009}.

%----------------------------------------------------------------------------------------
%	SUBSECTION: TYPES OF TASKS
%----------------------------------------------------------------------------------------

\subsection{Types of tasks}

All nodes in a behavior tree share certain similarities.
At their core, tasks are small execution units which are run for a certain length of time by the program which executes the behavior tree.
At the end of execution, the task returns a status code, typically success, failure, or running.
There are three types of tasks commonly used in behavior trees, \emph{actions}, \emph{conditions}, and \emph{composites}.

\subsubsection{Actions}

Actions are behaviors that change the state of the system.
For example, an action may cause the Cyborg to move to a given location or check the status of its battery.
Actions can be long running, and will return a running status code to the calling function while they are performing their task.

\subsubsection{Conditions}

Conditions are checks made to test a property of the system.
These can be used to do checks before executing an action, for example to check if the action is wanted or possible.
To continue the example given above, the battery status that was updated may be compared to a critical value.
In this way, an action can be executed if and only if it is found that the battery state permits it, or the system may execute behavior to recharge the battery if not.

\subsubsection{Composites}

\begin{listing}
    \inputminted{python}{\rootfolder/Chapters/Chapter2/Listings/priority.py}
    \caption{An example implementation of the Priority node in Python.}
    \label{lst:priority}
\end{listing}

Composites are nodes which tie the actions and conditions together.
While actions and conditions are implemented by the programmer, there are a fixed and very small set of typical composite nodes.
Generally there are two composite nodes that are used, the \emph{Priority} node and the \emph{Sequence} node.

In the simple example shown in~\cref{fig:behaviortree} the first control node, with the \emph{?} symbol, is a Priority node.
This node will execute, or tick, its children from left to right until one of the children returns successfully, as shown in~\cref{lst:priority}.

\begin{listing}
    \inputminted{python}{\rootfolder/Chapters/Chapter2/Listings/sequence.py}
    \caption{An example implementation of the Sequence node in Python.}
    \label{lst:sequence}
\end{listing}

The first child of this node, with the \emph{\rightarrow} symbol, is a Sequence node.
This node will execute its children from left to right until one of the children returns failure, as shown in~\cref{lst:sequence}.
By combining these, the tree shown in~\cref{fig:behaviortree} will first attempt to tick nodes \emph{t1} and \emph{t2}, and only if either of those fail will it tick node \emph{t3}.

This approach can be repeated to create increasingly complex behavior, without consideration of the actual behavior performed by the tasks \emph{t1}, \emph{t2}, and so on.
If behavior reuse is required, it is possible to compose the behavior in a sub-tree which can be included as a task in a parent tree.

%----------------------------------------------------------------------------------------
%	SUBSECTION: THE BLACKBOARD
%----------------------------------------------------------------------------------------

\subsection{The blackboard}

In order to create more complex behavior, it is generally necessary to be able to share data between tasks in a tree.
For example, the action which queries for the Cyborg's battery charge level would need somewhere to store this value, so that it can be used by the condition which compares the charge level to a critical value.
The \emph{blackboard}, as it is commonly called, is a data store used for this purpose.
Generally, a blackboard is a key-value store for data exchange between tasks.
The blackboard allows for storage of data that is available to any node in the tree.
Depending on the implementation, it may also allow for data to be available only to other nodes in a sub-tree or data that is only available to the node which stored the data.

%----------------------------------------------------------------------------------------
%	SUBSECTION: STRENGTHS AND WEAKNESSES
%----------------------------------------------------------------------------------------

\subsection{Strengths and weaknesses}

When using behavior trees, complex behavior arises from composition of simple behavioral components.
Where the state machine approach may encourage the developer to encode a large amount of behavior in each state, when using behavior trees it is useful to break tasks into their smallest useful parts.
By combining the behavior tree approach with a graphical interface for composing the tree, this allows for non-programmers to create trees from basic building blocks without needing to write code.

One commonly encountered limitation of behavior trees is that they make it difficult to implement state-based behavior.
For example, while it is simple to encode behavior that causes the Cyborg to dock with its charging station if the battery runs low, it is more difficult to make the Cyborg generally act in a way that conserves energy if it is in a state of having a somewhat low battery.
Doing so would require either that two similar behavior trees are implemented, one for a normal state and one for an energy-depleted state, or that each task in the tree is a state machine with these two states.
In this way, behavior trees share some of the weaknesses found in state machines.
While this is possible to implement this functionality, it adds significant design complexity and is therefore generally avoided~\cite{Millington2009}.

%----------------------------------------------------------------------------------------
%	SUBSECTION: IMPLEMENTATION IN ROS
%----------------------------------------------------------------------------------------

\subsection{Implementation in \acrshort{ros}}

Unlike finite state machines, \acrshort{ros} does not have a go-to implementation like \acrshort{smach}.
Therefore, to reach the objective of this thesis different approaches are evaluated.
These will be described in the following chapter.

\end{document}

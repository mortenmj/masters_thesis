% Chapter Template
\providecommand{\rootfolder}{../..} % Relative path to main.tex
\documentclass[\rootfolder/main.tex]{subfiles}
\begin{document}

\chapter{Background}
\label{ch:background}

%----------------------------------------------------------------------------------------
%	SECTION: HARDWARE
%----------------------------------------------------------------------------------------

\section{Hardware}

%----------------------------------------------------------------------------------------
%	SUBSECTION: PIONEER LX
%----------------------------------------------------------------------------------------

\subsection{Pioneer LX}

\begin{wrapfigure}{R}{0.4\columnwidth}
    \centering
    \subcaptionbox{NTNU Cyborg.\label{fig:robot-ntnu}}[0.38\columnwidth][l]{\pimage[0.38]{Figures/final-transp}}\par\bigskip
    \subcaptionbox{MobileRobots Pioneer LX.\label{fig:robot-original}}[0.32\columnwidth][r]{\pimage[0.32]{Figures/robot}}
    \caption{Illustration of the two interface types specified in the FMI standard.\label{fig:fmi}}
\end{wrapfigure}

The NTNU Cyborg uses the Pioneer LX robot, from MobileRobots.
Omron Adept MobileRobots, previously MobileRobots, is a manufacturer of intelligent mobile robots for commercial and industrial use.

The Pioneer LX comes equipped with the required sensors for safe, autonomous operation.
The robot is equipped with SICK S300 \ang{270} laser rangefinder for navigation and object detection, in addition to a SICK TiM 510 laser for frontal sensing near floor level.
Additionally, the robot is equipped with rear facing ultrasonic sonar sensors and front bumpers for collision detection.
These sensors allow for simultaneous mapping and localization (SLAM), which allows the robot to navigate without pre-mapping the environment.
The robot is designed for continuous operation for 13 hours before recharging, which can be performed autonomously.

%----------------------------------------------------------------------------------------
%	SUBSECTION: STEREOLABS ZED CAMERA
%----------------------------------------------------------------------------------------

\subsection{Stereolabs ZED camera}

\begin{figure}
    \pimage{Figures/zed}
    \caption{Stereolabs ZED}
    \label{fig:zed-camera}
\end{figure}

The ZED camera from Stereolabs is a dual 4 megapixel camera
It provides 1920x1080 video at 30 frames per second, or 800x400 at 100 frames per second, with a \ang{110} angle of view.

The ZED camera is designed to emulate the stereoscopic operation of human eyes.
By recording an image from two slightly offset points, depth and motion in space can be inferred by comparing the displacement of pixels in the left and right images.
This allows for the camera to provide a depth map of the recorded scene, where pixels are defined by an (x, y, z)-tuple rather than the normal (x, y) coordinate pair.
The information is exported as a black and white depth map.
By identifying the pixels of an object, the distance to the object can be determined by calculating the mean intensity of these pixels in the depth map.

%----------------------------------------------------------------------------------------
%	SECTION: SOFTWARE
%----------------------------------------------------------------------------------------

\section{Software}

The robot's onboard computer comes preinstalled with Ubuntu Linux and ROS.
MobileRobots provide a number of software packages for efficient use of the robot, among them ARIA, ARNL, MobileSim and MobileEyes.

%----------------------------------------------------------------------------------------
%	SUBSECTION: ROS
%----------------------------------------------------------------------------------------

\subsection{ROS}

ROS or the Robot Operating System, is an open-source framework for writing robot control software.
ROS consists of a collection of libraries and tools that simplify the task of implementing complex and robust control software for different robot platforms.
It is not an operating system in the traditional sense of managing and scheduling processes but acts as middleware, providing a communications layer, for a distributed control system.
Individual modules are known as nodes, and implement functionality such as low-level drivers, mapping and navigation, image perception, reasoning and decision-making and so on.

The framework provides middleware for individual software modules, known as nodes, so that modules from various sources can interact using a standard interface.
These can all be implemented separately and can communicate with each other using the ROS communications layer.
By allowing the control software to be modularized, ROS simplifies the development of robotics software.
This approach simplifies development, by allowing components developed by separate individuals, teams or organizations to communicate using a standardized interface.
This promotes collaboration and code reuse, allowing different research teams to focus on a subset of robot control rather than building a complete solution from the ground up.

We will give a run-down of the various concepts employed by ROS to enable this.

The components of ROS are \emph{nodes}, \emph{messages}, \emph{topics} and \emph{services} \cite{Quigley2009}.

\subsubsection{Nodes}

Nodes are processes that perform computation, and are primarily implemented in Python or C++.
Nodes are self-contained processes, and communicate with other nodes by passing messages through the ROS middleware.
ROS modules are generally constructed from a number of nodes.
This approach can be said to follow the Unix philosophy of "do one thing and do it well", which simplifies software development by limiting the scope of each individual node.
ROS allows for a control system to be divided into any number of nodes, which may run on the same machine or on different machines.

When implementing nodes in C++, it is also possible to implement them as nodelets.
A nodelet is implemented similarily to a normal node, but is instantiated by a special node, the nodelet\_manager.
This can be done by creating a separate node, which loads all the required nodelets, or it can be done using a launch file.

\subsubsection{Messages}

The information passed between nodes is encapsulated in messages, which may be thought of as analogous to structs in C.
Messages are composed of standard data types such as integers, floats and so on, and arrays of these.
A message can also be composed of other messages.
Many standard message formats are available in ROS, and it is also possible to define custom message types.
Nodes publish messages in one of two ways, either by broadcasting them to a topic or by answering service requests.

\subsubsection{Topics}

Messages can be published to a topic, which can be subscribed to by other nodes.
This publish-subscribe model allows for sharing of data in a broadcast manner, and there may be any number of publishers and subscribers for a topic.
One example of such a node is the feedback node, from move\_base, which publishes the current position of the robot.
Any other node which wants to know the current position can subscribe to this, and will receive the new position whenever the robot moves.

\subsubsection{Services}

While the publish/subscribe model works well for many types of information, it is necessarily an asynchronous form of communication.
For applications where synchronous communication is required, it is possible to use services.
A service is defined by two messages, the input message and the result message.
When called, the service will execute and the result is returned to the caller.
Services can be used to request information, request the execution of a physical action or some other task.
\cref{lst:ros_service} shows one such service, in use by the Cyborg project.

\begin{listing}
    \inputminted{python}{\rootfolder/Chapters/Chapter2/Listings/distance_to_goal.py}
    \caption{Example of an ROS service, written by the author.}
    \label{lst:ros_service}
\end{listing}


\subsubsection{Actions}

While service calls are useful for remote procedure calls which execute quickly, they are blocking and should be avoided for long-running tasks, or tasks that may have to be preempted.
For this purpose, ROS provides actions.
Actions are used for procedure calls that cause the robot to perform a long-running task, such as moving to a location or some other real-world action.
Actions are able to keep state for the lifetime of a provided goal, and will provide feedback to each client that issues a goal to the server.

\subsubsection{ARIA}

ARIA, or Advanced Robot Interface for Applications, is a C++ library for all robots from MobileRobots.
The library allows for dynamic control of the robot's velocity, heading, relative heading as well as other parameters.
These can be controlled both using a low-level interface and a higher-level Actions infrastructure.
ARIA receives sensor data such from the robot platform, such as position estimates, sonar data and laser rangefinder data.
While written in C++, the ARIA library can be accessed from other languages including Python.

\subsubsection{ARNL}

ARNL is a software package built on top of ARIA, which provides intelligent navigation and localization.
This provides information about the current position of the robot, and provides an interface for requesting that the robot move to a given location.
The software updates the current position automatically, using data from the robot's sensors and map.

These features are provided in ROS as a node, called ros-arnl, which exposes this functionality in the form of topics and services.
This node provides a simple interface for higher-level software to monitor and control the position of the robot.

\subsection{MobileEyes}

MobileEyes is a graphical interface from MobileRobots for monitoring robot motion and sensor output.
The program allows the user to monitor the movement of the robot on the map, and it is possible to send commands to the robot remotely.
The software also allows for reconfiguring the robot, and it is possible to send custom commands or create custom overlays that are shown on the map.

\subsection{MobileSim}

MobileSim is a simulation software package from MobileRobots, which allows for testing ROS modules in simulation.
The software emulates the physical robot such that other parts of the ROS integrate without any necessary changes.
This simulation includes data streams from sensors such as the sonar and laser rangefinders, which allows for efficient testing of control software during development.

%----------------------------------------------------------------------------------------
%	SECTION: BIOLOGICAL NEURAL NETWORKS
%----------------------------------------------------------------------------------------

\section{Biological neural networks}

\begin{figure}
    \pimage{Figures/biological-neuron}
    \caption{Model of a biological neuron TODO: SOURCE.\label{fig:biological_neuron}}
\end{figure}

As a long-term goal, the Cyborg project hopes to use biological neurons in a robot control loop.
Here, we will give some background on relevant biological concepts.
This information is largely from \cite{Knudsen2016}, where these concepts are explained in further detail.

Neurons are electrically excitable cells, which process and transmit information using electrical and chemical signals.
The signals travel via synapses, which are specialized connections between neurons.
The sum of neurons and their connections are referred to as a neural network, these are the core components of the central nervous system as well as the ganglia of the peripheral nervous system.

\begin{wrapfigure}{R}{0.5\columnwidth}
    \pimage[0.49]{Figures/brain-hierarchy}
    \caption{Model of the hierarcical structure of the brain\cite{Perry1999}.\label{fig:brain-hierarchy}}
\end{wrapfigure}

When the membrane potential of a neuron is excited past a certain threshold, an action potential (AP) is triggered.
An action potential is a rapid rise and fall in the membrane potential of the neuron, which causes the axon hillock to fire an electrical signal down the axon of the neuron.
The membrane potential of the neuron is excited by the firing of upstream neurons, and other extracellular potential changes.

A synapse is a structure which connects two neurons, and allows for the transmission of an electrical or chemical signal.
Synaptic communication generally travels along axons and synaptic terminals of the upstream neuron, to the dendrites of the downstream neuron.
Synapses may either transmit electric signals directly, in \emph{electrical synapses}, or using neurotransmitters, in \emph{chemical synapses}.

The neurons in the brain are organized in a hierarchical manner.
Signals enter through the brainstem, and travel upwards as shown in \cref{fig:brain-hierarchy}.
As shown, higher layers of the brain are responsible for increasingly sophisticated levels of thought.
This layering is also present in the neocortex.
Taking vision as an example, lower levels of the neocortex are responsible for simple features such as edges and corners.
Low-level patterns are combined at mid-levels into more complex features such as curves and textures.
Finally, at higher levels of the neocortex, complex objects such as cars and houses are recognized.

%----------------------------------------------------------------------------------------
%	SECTION: ARTIFICAL NEURAL NETWORKS
%----------------------------------------------------------------------------------------

\section{Artificial neural networks}

\begin{wrapfigure}{R}{0.4\columnwidth}
    \iimage[0.39]{Figures/artificial-neuron}
    \caption{Model of an artificial neuron.\label{fig:artificial_neuron}}
\end{wrapfigure}

Our robot is intended to move about its surroundings, and is equipped with a stereo camera in order to orient itself.
Here, an overview will be given of the techniques used in computer and robot vision, and the research that has driven the explosion of activity within this field in the past few years.
This section will provide some background on how artificial neural networks as they relate to robot vision, as well as an overview of how they function in general.

Artificial neural networks are an approach to complex computational tasks as an emergent process of a large number of simple interconnected units.
This approach is inspired by the activity of neurons in the brain.
The artificial neuron is modelled by an activation function, which outputs a number as a function of the sum of its inputs.
Historically, this has been one of several possible sigmoid functions, for example $g\left(x\right) = \frac{1}{1 - e^{-x}}$.
An illustration of an artificial neuron is shown in \cref{fig:artificial_neuron}.

Nodes in artificial neural networks are organized into layers of units which are connected to the units in the consequent and previous layers.
These layers are typically fully connected.
The value of the signal flowing into a node is a function of the value flowing out of the previous layer, and the weights assigned to the particular connections.
A neural network with a sufficient number of units and a continuous, bounded and non-constant activation function, is able to approximate any mathematical function\cite{Cybenko1989}\cite{Hornik1991}.

\begin{wrapfigure}{R}{0.4\columnwidth}
    \iimage[0.39]{Figures/xor-net}
    \caption{XOR network, illustrating how neurons can implement basic logic functions.\label{fig:xor_net}}
\end{wrapfigure}

A simple example, which approximates the XOR function, is illustrated in \cref{fig:xor_net}.
The weights in this example are determined by construction.
In a real scenario, the weights are found by minimizing some cost function through the process of gradient descent, which is referred to as training the network\cite{Mitchell1997}.
Through training, the network would find a different set of weights while still achieving the same output.

%----------------------------------------------------------------------------------------
%	SUBSECTION: CONVOLUTIONAL NEURAL NETWORK (CNN)
%----------------------------------------------------------------------------------------

\subsection{Convolutional Neural Network (CNN)}

\begin{figure}
    \iimage{Figures/mnist-net}
    \caption[An example of a neural network with one fully connected hidden layer]%
        {An example of a neural network with one fully connected hidden layer. %
         The input to the network is a 28$\times$28 pixel image of a single digit, flattened to a 784 element vector. %
         The output is a confidence score for each of the possible digits. %
         The network was trained by the author, and achieved 92.4\% verification accuracy on the MNIST handwritten digit dataset. %
         More complex networks exceed 99\% accuracy on this dataset \cite{mnist2010}.\label{fig:mnist-net}}
\end{figure}

\begin{figure}
    \pimage{Figures/features}
    \caption{Structure of a Convolutional Neural Network \cite{Mathworks}.\label{fig:cnn-classification}}
\end{figure}

For a fully connected network as the one shown in \cref{fig:mnist-net}, every neuron in the hidden layer is connected to each input.
In this case this is a mere 784 connections, one for each pixel.
However, with one input per color channel, per pixel, this number would quickly balloon into the hundreds of thousands for a larger image.
Instead of flattening an image to a column vector, as in the fully connected network shown before, a CNN arranges its neurons in a 3D volume corresponding to the width, height and color channel depth of the input image.
Furthermore, each neuron in a hidden layer is connected only to a smaller region of the previous layer, as shown in \cref{fig:cnn-classification}.
Between convolutional layers, pooling layers have traditionally been inserted to reduce the spatial size of the network.
This reduces the computational complexity of the network, and by reducing the number of parameters also limits overfitting.

\begin{figure}
    \pimage{Figures/abstract-features}
    \caption{Neural network learning increasingly abstract features \cite{Brown2015}.\label{fig:abstract-features}}
\end{figure}

A CNN consists of convolutional, pooling and fully connected layers, which perform the following tasks.
A convolutional layer consists of a set of learned filters.
These filters are moved across the input image, and as the network is trained it learns filters that activate when presented with some visual feature.
As shown in \cref{fig:abstract-features}, early layers may learn basic features such as horizontal or vertical lines and so on.
Later layers learn increasingly high-level features, such as the shape of an eye, the corner of a mouth or some other building block of an image.

%----------------------------------------------------------------------------------------
%	SUBSECTION: IMAGE CLASSIFICATION
%----------------------------------------------------------------------------------------

\subsection{Image classification}

\begin{figure}
    \iimage{Figures/ilsvrc}
    \caption[The influence of deep learning on image classification error rates.]%
            {Progress of image classification, and the growth of deep networks. %
             The graph shows the top-5 classification error of each year's ILSVRC winner, and the depth of the network used. %
             The 2011 winner did not employ a neural network solution
             \cite{Krizhevsky2012}\cite{Zeiler2013}\cite{Szegedy2014}\cite{He2016}.\label{fig:ilsvrc}}
\end{figure}

Image classification refers to the task of assigning images to one of a set of possible classes, based on the contents of the image.
Several such competitions are hosted yearly, where research teams compare progress in detecting a wide variety of objects.
While neural networks have been known since they were first described in literature in 1943\cite{Mitchell1997}\cite{Mcculloch1943}, their utility in complex tasks such as this has only begun to be developed in the past few years.
This development has largely been attributed to advances in available computing power, the size of available data sets and algorithmic advances which allow for training larger neural networks.

As explained in \cite{Krizhevsky2012}, the ability to correctly classify complex images into one of several thousand possible categories requires a model with a large learning capacity.
However, even with the millions of images in a dataset such as ImageNet it is infeasible to train an ordinary neural network to perform this task.
Convolutional Neural Networks (CNN) enable networks to detect features in images, with a lower computational complexity than standard nets.
At the most basic levels of the network, such features may be simple horizontal and vertical lines while later layers may learn more abstract features.
This is illustrated in \cref{fig:abstract-features} and \cref{fig:cnn-classification}.
Compared to ordinary feedforward networks, such as the one illustrated in \cref{fig:mnist-net}, CNNs have fewer connections and parameters and are therefore easier to train.

\begin{wrapfigure}{R}{0.4\columnwidth}
    \iimage[0.39]{Figures/activation}
    \caption{Illustration of Sigmoid and ReLU activation functions.\label{fig:activation-functions}}
\end{wrapfigure}

Until 2012, the top performing algorithms in the yearly ImageNet Large Scale Visual Recognition Challenge (ILSVRC) was dominated by algorithms requiring a large amount of manual hand coding of features, and that still had an error rate of over 26\%.
In 2012, researchers from the University of Toronto presented a deep CNN, made possible by using a new activation function termed the ReLU, see \cref{fig:activation-functions}.
The while the Sigmoid function quickly saturates, the ReLU function does not.
This makes networks using ReLU less susceptible to disappearing gradients, which has allowed for the training of a deeper network than previously possible\cite{Krizhevsky2012}.

The current revolution in deep learning started when researchers from the University of Toronto proposed a deep convolutional neural network using ReLU, at the 2012 ILSVRC.
Their network was able to classify images with an error rate of 15.3\%, outcompeting previous winners by over 10 percentage points, as shown in \cref{fig:ilsvrc}.
While the concept of convolutional neural networks (CNN) and deep learning had been known for many years, this was the first significant use of such networks in computer vision.

Later advances have improved the applicability of even deep neural networks further.
In 2015, the authors of \cite{He2016} proposed a novel technique of feed-forward from the input signal, leading to great improvement of the trainability of extremely deep networks.
As explained, a neural network can approximate any mathematical function.
The key insight by the authors is that if a network of a given depth can approximate such a function, and a functionally identical deeper network can be created by inserting unity layers, then a deeper network should not yield poorer performance than the original network.
However, the result found in practice was that as network depth increases, network performance flattens out and eventually degrades.
The authors proposed that this is a problem of training the network, rather than a fundamental problem of extremely deep networks.
To overcome this, the authors note that the desired mapping is likely to be closer to its input $\vec{x}$, than a zero mapping.
If the original desired mapping is the function $\mathcal{H}(\vec{x})$, the network is instead trained to approximate $\mathcal{F}(\vec{x}) = \mathcal{H}(\vec{x}) - \vec{x}$.
This greatly improves the trainability of the network and allows for the increase in network depth shown in \cref{fig:ilsvrc}.
Unlike previous approaches, the network achieves high accuracy through a deep but simple and repeating architecture composed of residual blocks as shown in \cref{fig:residual-block}.
This approach gives the network a computational complexity much lower than the number of layers might seem to imply \cite{He2016}.

\begin{wrapfigure}{R}{0.4\columnwidth}
    \pimage[0.39]{Figures/resnet}
    \caption{Residual block \cite{He2016}.\label{fig:residual-block}}
\end{wrapfigure}

As deeper networks are able to capture more abstract features in images, machine vision becomes more robus and consequently applicable to a larger range of tasks.
The organizers of ILSVRC have announced that the 2018 competition will involve classifying 3D objects as well as 2D images.

%----------------------------------------------------------------------------------------
%	SUBSECTION: OBJECT DETECTION
%----------------------------------------------------------------------------------------

\subsection{Object Detection}

In object classification tasks, the algorithm is trained to identify one object in the image, or alternatively the principal object in the image for images with more than one object.
In object detection, the goal is to locate and classify all objects in the image, including identifying the boundary between objects and how they relate to one another.
Just as deep learning has revolutionized object classification, great strides have been made in object detection in recent years.
As these developments are highly relevant to the field of robotics and the Cyborg project, this section will give an overview of these developments.

The same networks that are used for classifying images with a single object can also be applied to classify individual objects within a more complex image.
The challenge in object detection is to identify these individual objects, so that they can be classified in the manner described in the previous section.

Similar to the progress seen in object classification, previous to the deep learning revolution the best-effort approaches used other techniques than neural networks \cite{Girshick2013}.
The very earliest attempts at object detection using CNNs employ a naive technique of moving a sliding window across the image.
Using windows of various sizes and classifying these sub-images one by one, it is possible to detect objects within an image composed of multiple objects.
However, this is a brute force approach and for fine-grained detection requires classifying thousands of windows per image.

\subsubsection{R-CNN}

\begin{figure}
    \pimage{Figures/rcnn}
    \caption{Object detection and classification using R-CNN \cite{Girshick2013}.\label{fig:r-cnn}}
\end{figure}

The approach taken by \cite{Girshick2013}, called Regions with CNN (R-CNN) extracts region proposals from the image by a process called selective search.
The selective search algorithm looks at the image through windows of different sizes, and identifies relevant regions by grouping adjacent pixels by color, texture or intensity.
The resulting groups are reshaped into bounding boxes, and the contents of each bounding box are fed to an image classifier.
If the contents of a box are successfully classified, the algorithm attempts to tighten the bounding box using a linear regression model.
The process is shown in \cref{fig:r-cnn}.
The approach has low error rate, but the connection of three different models leads to high complexity which makes the system difficult to train.

\subsubsection{Fast R-CNN}

While this approach of generating region proposals is significantly less computationally expensive than using sliding windows, it still requires that around 2000 region proposals are classified.
In a follow-up paper, Gerschick proposes an approach in two significant ways.
Firstly, rather than running individual regions of the image through the feature classifier one by one, features are computed on the entire image in a single pass to create a feature map, in a process called Region of Interest Pooling.
Then, the features for each region can be obtained by selecting the appropriate area from the pre-computed feature map.
Secondly, the three models used previously (region proposer, image classifier and bounding box regression model), are combined into a single model.
This allows for end-to-end training, which greatly improves trainability \cite{Girschick2015}.

\subsubsection{Faster R-CNN}

\begin{wrapfigure}{r}{0.4\columnwidth}
    \pimage[0.39]{Figures/faster-rcnn}
    \caption{Object detection using Faster R-CNN \cite{Ren2017}.}
    \label{fig:faster-rcnn}
\end{wrapfigure}

While the advances outlined greatly improve on the efficiency of object detection algorithms, they expose the region proposer as a significant bottleneck \cite{Ren2017}.
As outlined previously, selective search is used to generate region proposals, while a CNN is used to extract features, classify the image and compute a bounding box.
The work done by \cite{Ren2007} uses the features computed by the CNN discussed previously, and combines this with a separate CNN, called the Region Proposal Network.
By making use of the same feature map that is used to classify images, the authors enable essentially cost-free region proposals.
The network passes a sliding window over the feature map, and computes region proposals along with an objectness-score.
The objectness-score measures the probability that the region contains an object, and allows for selecting only the regions that meet some minimum threshold.

\subsubsection{Mask R-CNN}

\begin{figure}[H]
    \pimage{Figures/mask-rcnn}
    \caption{Pixel-level object detection using Mask R-CNN \cite{He2017}.\label{fig:mask-rcnn}}
\end{figure}

Later work extends this approach by detecting which pixels belong to each of the detected objects, as shown in \cref{fig:mask-rcnn} \cite{He2017}.
Here, in parallel to the region proposal network, a network is branched off which simultaneously computes a binary mask for each object.
The binary map identifies which pixels belong to the object, as shown in \cref{fig:mask-rcnn}.

\subsubsection{You Only Look Once}

\begin{figure}[H]
    \pimage{Figures/yolo}
    \caption{Illustration of object detection using the YOLO algorithm\cite{JosephRedmon}.\label{fig:yolo}}
\end{figure}

The network used by this project, named YOLO, takes a novel approach.
Rather than making a first pass to detect shapes, and a second pass to classify them, these operations are performed by a single neural network.
The image is divided into a grid, and the probability that each cell in the grid forms part of a bounding box, as well as the probability that the contents of the box belongs to an image class, is computed simultaneously.
These values are then combined to obtain a class-specific confidence score which encodes both the probability of the particular class appearing in the box, and how well the box fits the object.
This allows the network to reason globally about the contents of the image, where the R-CNN approach treats each selected shape separately from the others.
Furthermore, performing detection in a single pass provides greater speed \cite{Redmon2015}.

%----------------------------------------------------------------------------------------
%	SECTION: TRANSFER LEARNING
%----------------------------------------------------------------------------------------

\section{Transfer learning}

\begin{figure}
    \pimage{Figures/nvidia}
    \caption{Visualization of free-space detection and 3D object detection for autonomous driving \cite{NVIDIA}.\label{fig:nvidia-cnn}}
\end{figure}

The level of performance discussed in the previous section requires the training of very large convolutional neural networks.
Due to problems of overfitting in such large networks, it is necessary to train the network on very large datasets.
In \cite{NVIDIA}, researchers from NVIDIA describe a process by training such a network on the ImageNet dataset, consisting of approximately 1.2 million images.
These are in turn augmented by various transformations to a total dataset of 22 million images.
By training on such a large dataset, it is possible to create a very robust feature detector.
However, the amount of training required can take weeks or months, as in the example described by the researchers.

The features learned by a convolutional neural network, as shown in \cref{fig:abstract-features}, have been found to be similar across many different applications.
Observe in \cref{fig:image-classification}, that the majority of a network performs feature detection while only the last few layers use these features to classify the image.
One application of transfer learning is adapting an existing network to a new domain.
This can be done by replacing erasing the weights of the classification part of the network, or replacing these layers with new layers if required.
By keeping the weights of the feature learning layers of the network fixed, the network can be retrained to a new purpose.
In the case of the work done at NVIDIA, a general image classifier trained on the ImageNet dataset was repurposed to identify objects in the autonomous vehicle domain.
This can then be done much faster, and with a much smaller dataset, than what is required train a new network.
Using this technique resulted in a network able to detect pedestrians and vehicles, and identify which sections of the road are safe for driving, as shown in \cref{fig:nvidia-cnn}.

\section{Motivation}

\section{Functions of the robot}

\section{Related projects}

\section{Input and output}

\section{Objective}

%\chapter{Components}

%\label{ch:components}

%----------------------------------------------------------------------------------------
%	SECTION: FINITE STATE MACHINES
%----------------------------------------------------------------------------------------

\section{Finite State Machines}

\begin{wrapfigure}{R}{0.5\columnwidth}
    \pimage[0.49]{Figures/statemachine}
    \caption{An example of a simple state machine.}
    \label{fig:fsm}
\end{wrapfigure}

States are connected by transitions, which lead from one state to another.

%----------------------------------------------------------------------------------------
%	SUBSECTION: DEFINITION
%----------------------------------------------------------------------------------------

\subsection{Definition}

A finite state machine is an abstract model of computation, that allows for actions to be encoded into a finite number of states.
Changes from one state to another are referred to as transitions.
The transition from an initial state to a new state is governed only by the starting state, and the event, as shown in \cref{fig:fsm}.
Whenever the conditions for a transition are met, the system will perform that transition and execute the action associated with the new state.
State machines typically have actions associated with entering and exiting a state.

%----------------------------------------------------------------------------------------
%	SUBSECTION: EVENTS
%----------------------------------------------------------------------------------------

\subsection{Events}

The type of state machine described here is termed an event-driven state machine.
As the ROS system needs to perform many different actions, it is necessary that we not block the system by polling for new events.
Instead, events arrive asynchronously and are consumed by the state machine before it goes back to sleep.

%----------------------------------------------------------------------------------------
%	SUBSECTION: ACTIONS
%----------------------------------------------------------------------------------------

\subsection{Actions}

\section{Strengths and weaknesses}
\label{sec:state_machine_strength_weaknesses}

An overview of the use of state machines for AI in computer games is given in \cite{Millington2009}.
In general, the concerns outlined here overlap with the concerns relevant in the sort of high-level decision making that is necessary in the Cyborg project.

\subsection{Implementation in ROS}

The existing control system at the outset of this project was implemented using SMACH, a task-level state machine architecture for ROS.
SMACH allows for fast prototyping and implementing complex state machines.
As it is a task-level architecture, it is not suitable for low-level control but rather for high-level decision making.
As mentioned in \cref{sec:state_machine_strength_weaknesses}, state machines are not well suited to unstructured tasks.
This is also stated by the SMACH developers.

States in SMACH are implemented as individual classes, which allows for reuse.
This also allows for reuse of behavior between states, by using object composition.
Actions to be performed by the robot are associated with states in the state machine, and the robot will carry out that action for as long as it is in the particular state.

\begin{listing}
\inputminted{python}{\rootfolder/Chapters/Chapter2/Listings/smachstate.py}
\caption{State example from the SMACH documentation.}
\end{listing}

\section{Behavior trees}

\begin{wrapfigure}{L}{0.5\columnwidth}
    \pimage[0.49]{Figures/behaviortree}
    \caption{An example of a simple behavior tree.}
    \label{fig:bt}
\end{wrapfigure}

\subsection{Definition}

\subsection{Strengths and weaknesses}

\subsection{Implementation in ROS}

While several behavior tree implementations were considered for this project, in the end the choice was made to use a library named behavior3.

\subsection{Events}

\subsection{Actions}

%\chapter{Physical body}

%\section{Concurrent projects}

%\section{LED display}

%\section{Stereo camera}

%\section{Data streaming}

%\chapter{Pioneer LX}

%\section{Description}

%\section{Sensors}

%\section{Software}


%\section{MobileRobots}

%\section{Applications}

%\chapter{Visualization}

%\section{State machine}

%\section{Behavior tree}

%\section{Considerations}

%\section{Message format}

%\chapter{Material and Methods} % Main chapter title

\label{ch:matmet} % Change X to a consecutive number; for referencing this chapter elsewhere, use \ref{ChapterX}

\section{ROS}

As the Cyborg initially used the original version of ROS, and was ported to ROS 2 by the author, we provide some background on both.
The original version of ROS will be referred to as ROS, while the newer ROS 2 version will be referred to as such.

\subsection{Components}

\subsection{ROS 2}

Here, we will give an overview of ROS 2, and how the project has attempted to remedy some perceived shortcomings in ROS.

As outlined in \cite{Gerkey2017} ROS was developed based on the use case of the Willow Garage PR2 robot.
Due to this, development was guided by some considerations that have later shown to be detrimental, including

\begin{itemize}
    \item a single robot
    \item significant computational resources
    \item no real-time requirements
    \item strong network connectivity
    \item mainly academic applications
\end{itemize}

Since the beginning of ROS in 2007, several of these assumptions have changed, and ROS has been used on a far wider range of robots than for what it was originally designed.

Among the new use cases outlined, are

\begin{itemize}
    \litem{Control of multiple robots}
        Currently, there is no standard way to control more than a single robot using ROS.
        ROS has a single-master architecture, and multi-robot support does not elegantly integrate into this design.
    \litem{limited computational resources}
        ROS is not designed to run on microcontrollers. Therefore, nodes must interact with these through a device driver.
        ROS 2 is designed so that these controllers can be implemented as nodes, and thereby participate directly in the control system as first-class citizens.
    \item{built-in real-time support}
    \litem{non-ideal networks}
        As seen in earlier work on this project by \cite{Waløen2017}, ROS does not degrade gracefully when run on unreliable networks.
        ROS 2 aims to alleviate this.
    \litem{academic and industrial applications}
        ROS 2 aims for ROS to remain the platform of choice in academic robotics, while also becoming increasingly relevant in industrial applications.
\end{itemize}

The central feature of ROS is the publish-subscribe middleware which allows for loose coupling of individual nodes.
As ROS was begun in 2007, there was no sufficiently mature off-the-shelf technology that provided this, and this system was built essentially from scratch.
The ROS project implemented the necessary framework for node discovery, message definition, serialization and transport.
Since 2007, a number of technologies have mature which provide this capability, and it would not have been necessary to build a custom solution today.
Several advantages to this are listed by \cite{Gerkey2017}:

\begin{itemize}
    \item Less code to be maintained by the project developers
    \item Third party solutions may offer features outside the scope of what the project could develop themselves
    \item The project benefits from ongoing improvements to third party solutions
    \item Third party solutions may be rigorously proven, and thereby improve the perceived reliability of ROS
\end{itemize}

\section{Decision Making}

At the outset of the work on this thesis, the cyborg was running the decision making software outlined in \cite{Andersen2017}.

While state machines are a well-proven approach to flow control in many applications, there was a desire to investigate alternative approaches.
This section will provide some background on these two approaches.

\subsection{State Machines} \label{sec:statemachines}


\subsection{Behavior Trees} \label{sec:behaviortrees}


\section{Monitoring Application}

The main purpose of this paper is to create an application which allows the user to monitor the state of the robot.
This will include showing a graphical representation of the behavior tree, showing the control flow as the tree is running.
If possible, it will also include a representation of the map the robot uses to navigate, and real-time visualization of sensor data.
It will also include the ability to interact with the robot, such as sending it to a particular location.

\end{document}

% Chapter Template
\providecommand{\rootfolder}{../..} % Relative path to main.tex
\documentclass[\rootfolder/main.tex]{subfiles}
\begin{document}

\chapter{Introduction} % Main chapter title

\label{ch:introduction} % Change X to a consecutive number; for referencing this chapter elsewhere, use \ref{ChapterX}

\section{NTNU Cyborg}

NTNU Cyborg is a project to develop a cybernetic organism.
The main goal of the project is to enable communication between live nerve tissue and a robot.
The robot acts as a research platform for the study of neural signals, robotics and cybernetic machines.
The stated intention of the project is to bring NTNU to the forefront of international research in these areas.

\subsection{Neuroscience}

The cyborg has a biological "brain" which consists of a collection of nerve cells grown over a Micro-Electrode Array (MEA).
These spontaneously organize into neural networks, and communicate with each other using electronic impulses.
The impulses are captured by the MEA, and allows for streaming of nerve activity data to the robot.
The intent of this project is to analyze the electrical input and stimulate the network, in order to form a closed-loop system which allows the network to be trained.

\subsection{Robotics}

The cyborg has an autonomous robotic body, consisting of a commercial research robot as well as additions designed at NTNU.
The intent of the project is to implement a software control system for this robot to enable it to roam the hallways of the university, and interact with the people it meets.
The robot base is also intended to display a representation of the measured nerve activity.

\section{Previous work on the Cyborg project}

\section{Motivation and goal}

\end{document}

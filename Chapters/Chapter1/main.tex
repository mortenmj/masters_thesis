% Chapter Template
\providecommand{\rootfolder}{../..} % Relative path to main.tex
\documentclass[\rootfolder/main.tex]{subfiles}
\begin{document}

\chapter{Introduction} % Main chapter title

\label{ch:introduction} % Change X to a consecutive number; for referencing this chapter elsewhere, use \ref{ChapterX}

\section{NTNU Cyborg}

NTNU Cyborg is a project to develop a cybernetic organism.
The main goal of the project is to enable communication between live nerve tissue and a robot.
The robot acts as a research platform for the study of neural signals, robotics and cybernetic machines.
The stated intention of the project is to bring NTNU to the forefront of international research in these areas.

\subsection{Neuroscience}

The cyborg has a biological "brain" which consists of a collection of nerve cells grown over a Micro-Electrode Array (MEA).
These spontaneously organize into neural networks, and communicate with each other using electronic impulses.
The impulses are captured by the MEA, and allows for streaming of nerve activity data to the robot.
The intent of this project is to analyze the electrical input and stimulate the network, in order to form a closed-loop system which allows the network to be trained.

\subsection{Robotics}

The cyborg has an autonomous robotic body, consisting of a commercial research robot as well as additions designed at NTNU.
The intent of the project is to implement a software control system for this robot to enable it to roam the hallways of the university, and interact with the people it meets.
The robot base is also intended to display a representation of the measured nerve activity.

\section{Previous work on the Cyborg project}

Previously, student groups have added a rudimentary body for the robot, upon which the existing LED box is mounted.
This also acts as a mounting frame for existing equipment, including the power supply, Jetson computer, Raspberry Pi computer and so on.
There is also a stereo camera mounted on the robot, although it is currently not in use.
Work has also been done to pass information from the Micro-Electrode Array to the existing LED box, but this is also not in use at the outset of this project.

\section{The author's contribution}

This project aims to complete the ROS-based state machine, by implementing a variation on traditional state machines called a behavior tree.
This will involve integrating a behavior tree implementation, ideally one that already exists and is well tested, with the ROS architecture.
Furthermore, it will involve developing computer program for visualization of the behavior tree so that the state of the behavior tree can be inspected during robot operation.
The mounting frame that is already on the robot will be improved by mounting industry-standard 10 inch rack-mounting rails to the frame.
This will allow for secure, modular mounting of hardware within the robot body.
Finally, imagery from the stereo camera will be analyzed to detect objects in the robot's view field, such as humans and other everyday objects.
These will be classified by a neural network, and the distance to each object will be calculated so their location relative to the robot can be determined.

The goal of these tasks is to take the robot a few steps closer to a state where it is able to autonomously roam the hallways of the university and act as a mascot for the university and the Cyborg project.

Several existing software applications are used to achieve these goals, and will be described in more detail in the following chapter.
The software used by the author is primarily ROS, or the Robot Operating System.
ROS is a middleware system, composed by a number of modules, for the sake of this thesis primarily those delivered by MobileRobots, manufacturers of the Pioneer LX.
The ROS modules, or nodes, developed by the Cyborg project and by the author interface with these existing modules.
A library called b3, for implementing behavior trees, has been used to add a control system to our ROS environment.
Visualization of our behavior tree was developed as a module, or plugin, which extends the rqt software, a graphical interface for ROS.
Object detection was implemented using the OpenCV library.

\end{document}

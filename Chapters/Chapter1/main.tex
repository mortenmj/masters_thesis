% Chapter Template
\providecommand{\rootfolder}{../..} % Relative path to main.tex
\documentclass[\rootfolder/main.tex]{subfiles}
\begin{document}

\chapter{Introduction} % Main chapter title
\label{ch:introduction} % Change X to a consecutive number; for referencing this chapter elsewhere, use \ref{ChapterX}

\emph{\acrshort{ntnu} Cyborg} is a project to develop a cybernetic organism.
The main goal of the project is to enable communication between live nerve tissue and a robot.
The robot acts as a research platform for the study of neural signals, robotics and, cybernetic machines.
It is the stated intention of the project to bring \acrshort{ntnu} to the forefront of international research in these research areas.

This thesis contributes to this goal by implementing the following functionalities in the Cyborg:

\begin{itemize}
    \item A robust and extensible control system, based on behavior trees
    \item An application for visualizing the control system during execution
    \item An object detection and classification system using stereoscopic imagery and neural networks
\end{itemize}

In the following, the two main research areas for which the Cyborg serves as a platform are introduced, and the existing functionalities are summarized.

\subsection{Neuroscience}

The Cyborg has a biological \say{brain} which consists of a collection of nerve cells grown over a \acrfull{mea}.
These spontaneously organize into neural networks, and communicate with each other using electronic impulses.
The impulses are captured by the \acrshort{mea}, which allows for streaming of nerve activity data to the Cyborg.
The long-term intent of the neuroscientific part of the Cyborg project is to analyze the electrical input and stimulate the network, in order to form a closed-loop system which allows the network to be trained.
Further information on this research can be found in~\cite{Knudsen2016}.

\subsection{Robotics}

The Cyborg has an autonomous robotic body, consisting of a commercial research robot as well as additions designed at \acrshort{ntnu}~\cite{Waløen2017}.
The intent of the robotics part of the Cyborg project is to create a robot able to roam the hallways of the university, and interact with the people it meets.
The robot is intended to receive data from the \acrshort{mea} sensors described above, and display a representation of the measured neural activity.

\section{Previous work on the Cyborg project}

Previously, student groups have added a rudimentary body for the Cyborg, upon which the existing \acrshort{led} box is mounted.
This acts as a mounting frame for existing equipment, including the power supply, Jetson computer, Raspberry Pi computer and so on.
There is a stereo camera mounted on the Cyborg, although it is currently not in use.
Work has been done to pass information from the \acrlong{mea} to the existing \acrshort{led} box, but this is not in use at the outset of this thesis.

\section{Contributions of the author}

This thesis aims to enhance the Cyborg's control system, based on the \emph{\acrfull{ros}}, by implementing an alternative to traditional state machines called a \emph{behavior tree}.
This involves integrating a behavior tree implementation, ideally one that already exists and is well tested, with the \acrshort{ros} architecture.
Furthermore, it involves developing computer software for visualization of the behavior tree so that the state of the behavior tree can be inspected during Cyborg operation.
The mounting frame that is already on the Cyborg is improved by installing industry-standard 10 inch rack-mounting rails to the frame.
This allows for secure, modular integration of hardware within the Cyborg body.
Finally, imagery from the stereo camera is analyzed to detect objects in the Cyborg's view field, such as humans and other everyday objects.
These are classified by a neural network, and the distance to each object is calculated so their location relative to the Cyborg can be determined.

The goal of these tasks is to take the Cyborg a few steps closer to a state where it is able to autonomously roam the hallways of the university and act as a mascot for the university and the Cyborg project.

Several existing software applications are used to achieve these goals, and will be described in more detail in the following chapter.
The software used by the author is primarily the \acrfull{ros}.
The \arcshort{ros} software is run on the Cyborg's \emph{Pioneer LX} base, produced by \emph{MobileRobots}.
\acrshort{ros} is a middleware system, composed by a number of modules, for the sake of this thesis primarily those delivered by MobileRobots.
The \acrshort{ros} modules, or nodes, developed by the Cyborg project and by the author interface with these existing modules.
A library called \emph{behavior3}, for implementing behavior trees, has been used to add a control system to our \acrshort{ros} environment.
Visualization of our behavior tree is developed as a module, or plugin, which extends the \emph{rqt} software, a graphical interface for \acrshort{ros}.
Object detection is implemented using the \emph{OpenCV} library.

\end{document}

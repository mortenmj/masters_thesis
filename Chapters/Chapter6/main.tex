% Chapter Template
\providecommand{\rootfolder}{../..} % Relative path to main.tex
\documentclass[\rootfolder/main.tex]{subfiles}
\begin{document}

\chapter{Other tasks}
\label{ch:other-tasks} % Change X to a consecutive number; for referencing this chapter elsewhere, use \ref{ch:foo}

In addition to the major tasks presented previously, some smaller tasks have been performed in the process of writing this thesis.
These tasks are described in this chapter.

%----------------------------------------------------------------------------------------
%	SECTION RACK MOUNTING SYSTEM
%----------------------------------------------------------------------------------------

\section{Rack mounting system}

\begin{figure}[h]
    \centering
    \subcaptionbox{Before.\label{fig:before-rack}}[0.38\columnwidth][l]{\pimage[0.38]{Figures/before}}
    \subcaptionbox{After.\label{fig:after-rack}}[0.38\columnwidth][r]{\pimage[0.38]{Figures/after}}
    \caption{The Cyborg, showing equipment mounting before and after installation of the 10 inch rack.}
    \label{fig:before-after-rack}
\end{figure}

At the outset of the project, the Cyborg was equipped with a mounting frame, as shown in~\cref{fig:before-rack}.
A 10 inch rack mounting system was planned for this mounting frame, inspired by the use of 10 inch equipment enclosures used for home and small business networking equipment.
An example of such an enclosure is shown in~\cref{fig:rack-example}.

\begin{figure}[h]
    \subcaptionbox{Example of a 10 inch cabinet.\label{fig:rack-example}}[0.48\columnwidth][l]{\pimage[0.38]{Figures/rack-example}}
    \subcaptionbox{Rack rails fitted on the Cyborg.\label{fig:rack-detail}}[0.48\columnwidth][r]{\pimage[0.38]{Figures/rack-detail}}
\end{figure}

10 inch racks, or network enclosures, are a common equipment standard for small network equipment such as network patch panels, switches and so forth.
By using standard dimensions for enclosure width and height, as well as standard placement for mounting holes, equipment can easily be swapped in and out of the rack.
This modularity was found to simplify work on the Cyborg. 
Both the existing power supply and the enclosure for the Jetson TX2 computer were mountable in the rack, while leaving ample space for one or two extra modules in the future.
By fitting the Cyborg with four rack posts, as shown in~\cref{fig:rack-detail}, it is possible to mount the power supply facing backwards.
This enables neater routing of power cables, as most computing and networking equipment have power connectors at the back.
The Cyborg can be seen before and after installation of the new rack mounting system in~\cref{fig:before-after-rack}.
The rack itself was constructed by the authors of~\cite{Johansen2018}, as part of their Experts in Team project, with guidance from the author of this thesis.

\section{Convenience functionality for future ROS development}

To support collection functionality used by all nodes in the Cyborg software environment, two nodes were created called \emph{cyborg\_types}, \emph{cyborg\_util} and \emph{cyborg\_nav}.
These will be described in this section.

\subsection{cyborg\_types}

Common data types that are useful in multiple parts of the project are included in this node, so that they may be easily imported where needed.
Generally, these provide useful convenience functions for processing received \acrshort{ros} messages.

\subsubsection{Point}

The \emph{Point} class represents a point in space.
The data used to create a Point is generally received as a geometry\_msgs/Pose or geometry\_msgs/Pose2D message, both of which include a geometry\_msgs/Point message from the navigation subsystem in \acrshort{ros}.
In addition to exposing the x, y and z coordinates of the 3D point, the class also provides a convenience method for computing the Euclidean distance to a second Point.

\subsubsection{Quaternion}

The \emph{Quaternion} class represents an orientation in space.
The geometry\_msgs/Pose message includes a geometry\_msgs/Quaternion message, which contains the x, y, z and w coordinates used to instantiate the class.
The class also includes a method for instantiating from a geometry\_msgs/Pose2D message, which includes the orientation as an Euler angle.

\subsubsection{Pose}

The \emph{Pose} class represents a point and orientation in space, represented by the previous two classes.
In addition to providing class methods allowing the class to be instantiated from any of the Pose messages in geometry\_msgs, the class also provides a convenience method for computing the Euclidean distance to a second Pose.

\subsubsection{Path}

A \emph{Path} is an ordered collection of Poses, which typically represents a path between two locations in the world.
In the Cyborg control system, it is used to keep an updated representation of the current path the Cyborg is moving along, so that progress can be tracked.
To do this, it includes methods for adding and removing Poses from the Path, as well as advancing along the Path to a given Pose -- so long as the given Pose is within a certain distance from a Pose along the Path.
The class also contains a method to retrieve the total distance from the first to the last Pose along the Path.

\subsection{cyborg\_util}

cyborg\_util currently contains a single class, \emph{Locations}.
The Locations class retrieves a list of available goals from the \emph{AvailableGoals} service, which is described below.
The class acts as a persistent cache for the locations, exposing them as Pose objects.

\subsection{cyborg\_nav}

For simplifying navigational tasks, a set of services was created that can be used by e.g.\ nodes in the behavior tree.
These provide functionality which might be useful to multiple parts of the Cyborg software environment, and to simplify code reuse are created as individual services.

\subsubsection{AvailableGoals service}

For managing a set of known locations, and exposing these to other parts of the control system, a service was created called AvailableGoals.
The service is shown in~\cref{lst:availablegoals}.
The service is instantiated with a map file, which contains a list of named waypoints or goals.
Upon request of the list of available goals, the service will parse this map, and create a response message containing the list.
This list will be cached, so that it can be returned in the future without needing to recreate it.

\begin{listing}
    \inputminted[fontsize=\scriptsize]{python}{\rootfolder/Chapters/Chapter6/Listings/available_goals.py}
    \caption{Implementation of the AvailableGoals service.}
    \label{lst:availablegoals}
\end{listing}

\subsubsection{DistanceToGoal service}

For requesting the distance to a goal or waypoint measured from the current location of the Cyborg, a service was created called \emph{DistanceToGoal}.
As shown in~\cref{lst:distancetogoal}, the service will receive a goal, from which a Pose object is created.
The Pose class is included with the attached source code.
Using the created Pose object, the service calls the \emph{MakePlan} service, which is exposed by the ros\_arnl node from MobileRobots, described in~\cref{ch:background}.
The DistanceToGoal service will then receive a plan, consisting of a list of positions, which show the path from the Cyborg to the desired final position.
The list of positions is used to create a Path object.
The Path class, which is also included in the attached source code, acts as a wrapper for this list of positions, and exposes useful functions for interacting with the list.
The DistanceToGoal service will retrieve the length of the Path object, which is then returned to the user.
The length is found by calculating the Euclidean distance between each pair of coordinates along the path.

\begin{listing}
    \inputminted[fontsize=\scriptsize]{python}{\rootfolder/Chapters/Chapter6/Listings/distance_to_goal.py}
    \caption{Implementation of the DistanceToGoal service.}
    \label{lst:distancetogoal}
\end{listing}

\subsubsection{ClosestGoal service}

For requesting the closest goal, or waypoint, from the current location of the Cyborg, a service was created called \emph{ClosestGoal}.
Upon creation, this service instantiates a Location object.
The Location class is included in the attached source code.
When the Location class is instantiated, it will call the AvailableGoals service, and create a cache to hold the list of available goals.
The ClosestGoal service iterates through this list, calling the DistanceToGoal service for each available goal, and return the goal with the shortest distance from the Cyborg.

\begin{listing}
    \inputminted[fontsize=\scriptsize]{python}{\rootfolder/Chapters/Chapter6/Listings/closest_goal.py}
    \caption{Implementation of the ClosestGoal service.}
    \label{lst:closestgoal}
\end{listing}

\subsubsection{LoadMap service}

The ros\_arnl service LoadMap provides the ability to load a new map file to the Cyborg.
Unfortunately, the service does not allow for automatically loading a given map on startup.
In order to automatically load this map on startup, a simple service was created to call the ros\_arnl LoadMap service.
The Cyborg LoadMap service takes a single startup parameter, \emph{map\_file}, which is used to load the map using the ros\_arnl LoadMap service.
The service implementation is shown in~\cref{lst:loadmap}.

\begin{listing}
    \inputminted[fontsize=\scriptsize]{python}{\rootfolder/Chapters/Chapter6/Listings/load_map.py}
    \caption{Implementation of the LoadMap service.}
    \label{lst:loadmap}
\end{listing}

\end{document}

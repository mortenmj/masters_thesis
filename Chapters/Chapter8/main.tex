% Chapter Template
\providecommand{\rootfolder}{../..} % Relative path to main.tex
\documentclass[\rootfolder/main.tex]{subfiles}
\begin{document}

\chapter{Conclusion} % Main chapter title
\label{ch:conclusion} % Change X to a consecutive number; for referencing this chapter elsewhere, use \ref{ch:foo}

In summary, the following tasks have been planned and implemented:

\begin{itemize}
    \item a control system based on behavior trees
    \item a visualization application for the control system
    \item an object detection system using neural networks
\end{itemize}

Additionally, the author has guided and assisted \acrshort{eit} students contributing to the Cyborg.
The outcome of the major tasks has been described and discussed.

The behavior tree control system functions well, and is easier to extend than the state machine system used previously.
Both approaches have shortcomings in expressing multi-modal behavior.
This is expected, and is described in \cref{ch:background} and \cref{ch:results}.

The visualization application proved to be a useful debugging tool when creating behavior trees.
Extending this software to provide more interaction with the behavior tree would be useful, although technically challenging.

The object detection algorithm works well, but would benefit from improved \acrshort{gpu} support in the underlying library.

When developing software, it is easy to look back and criticize the architectural choices that have been made.
During the development of this thesis, revisiting previous code and design choices has been a continuous process throughout the work that has been carried out.
In the end this has been worth the effort, as it has led to robust new functionality implemented in the Cyborg, which yield considerable improvements to its control system.

\end{document}

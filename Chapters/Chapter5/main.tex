% Chapter Template
\providecommand{\rootfolder}{../..} % Relative path to main.tex
\documentclass[\rootfolder/main.tex]{subfiles}
\begin{document}

\chapter{Implementation of object detection and classification} % Main chapter title
\label{ch:implementation-objdet} % Change X to a consecutive number; for referencing this chapter elsewhere, use \ref{ch:foo}

To provide the Cyborg's control system with information about objects in the Cyborg's surroundings, an object detection and classification system was implemented.
This system uses data from the Cyborg's depth-sensing stereo camera to both identify and locate objects in the Cyborg's surroundings.
It continues the work done by \acrfull{eit} students working for the Cyborg project, who evaluated various object detection approaches and produced a demonstration using the \acrfull{yolo} algorithm \cite{Opheim2018}.
Throughout the time spent on this thesis, the author has collaborated with and guided this group in their work.
This has allowed the author to focus on proven ideas, unlike previous parts of this thesis where much effort was spent identifying promising solutions.
While partially based on these ideas, the following work is done by the author.
This chapter will provide details on the goals of this system, the choices that were made in planning the system and a description of the implementation of the system itself.

%----------------------------------------------------------------------------------------
%	SECTION: GOALS
%----------------------------------------------------------------------------------------

\section{Goals}

As described in~\cref{ch:background}, the ros-zed-wrapper from StereoLabs exposes the left and right images of the Cyborg's stereo camera, as well as a depth map calculated from these images.
Here, we intend to analyze the images from the stereo camera to detect objects in the images, and then calculate the distance to each object.
The output from the system will be twofold.
First, an image with bounding boxes drawn around each detected object is published as an \arcshort{ros} topic.
Furthermore, the object's class and distance from the Cyborg also drawn on the image.
Secondly, a list of detected objects, along with a header containing the time stamp of the prediction message, and a copy of the header from the input image is also published as an \acrshort{ros} topic.
Each detected object will be represented by a structure containing its classification, a polygon structure describing its bounding box, its distance from the Cyborg, and the classification confidence outputted by the detection algorithm.

\begin{listing}
    \inputminted{python}{\rootfolder/Chapters/Chapter5/Listings/Prediction.msg}
    \inputminted{python}{\rootfolder/Chapters/Chapter5/Listings/Predictions.msg}
    \caption{Prediction and Predictions message formats.}
    \label{lst:prediction-msg}
\end{listing}

%----------------------------------------------------------------------------------------
%	SECTION: EVALUATION OF ALTERNATIVES
%----------------------------------------------------------------------------------------

\section{Evaluation of alternatives}

As described in~\cref{ch:background}, there are many possible algorithms that perform object detection and image classification.
Key to this project is using an approach that performs well using the limited available computational resources on the Cyborg.
It was also desirable to use off-the-shelf software, to minimize the work needed to create and maintain the solution.

As described, the \acrshort{yolo} algorithm allows for object detection and classification using a single pass through the neural network, which is an efficient approach \cite{Redmon2015}.
As the object detection part of this thesis continued existing work by \cite{Opheim2018}, the choice was made to continue to use the \acrshort{yolo} algorithm.
Consideration was given to which implementation of this algorithm to use.

The demo implementation by \cite{Opheim2018} used a library called PyYOLO.
PyYOLO is a Python wrapper around the original CUDA implementation by the author of \acrshort{yolo}, and therefore performs very well.
The downside to this is that the underlying \acrshort{yolo} implementation must be compiled for PyYOLO to be used, which adds to the maintenance burden of the project members.

As of late 2017 there is support for \acrshort{yolo} in the \emph{\acrfull{dnn}} module of OpenCV, and it was decided that this merited further research.
While OpenCV lacks \emph{CUDA} support, and was therefore expected not to perform as well in timing benchmarks, this is planned for inclusion in future versions of OpenCV.
Furthermore, OpenCV's \acrshort{dnn} module is able to use a number of neural networks, which allows for different object detection algorithms to be run with only minimal changes to the Cyborg software.

%----------------------------------------------------------------------------------------
%	SECTION: IMPLEMENTATION
%----------------------------------------------------------------------------------------

\section{Implementation}

In this section, we will detail to separate implementations using this algorithm, written by the author.
These implementations make use of a reference implementation of the YOLO algorithm, to meet the goals outlined above, but are subject to different considerations.

%----------------------------------------------------------------------------------------
%	SUBSECTION: C++ IMPLEMENTATION
%----------------------------------------------------------------------------------------

\subsection{C++ implementation}

First, a C++ implementation using OpenCV was developed.
There are two significant advantages to implementing the object detection node in C++.
First, it is possible to run the detection algorithm as a nodelet in the same process as the ros-zed-wrapper.
A nodelet is a variation on the concept of an \acrshort{ros} node, as described in~\cref{ch:background}.
Rather than running the zed-ros-wrapper node and the object detection node in separate processes, which would require transporting raw image data over TCP, these are run in the same process.
This allows us to use the same publish/subscribe interface as in a normal node, but have message transport handled using shared memory.
This change is handled seamlessly by the Nodelet class, which nodelets inherit from.
The only caveat the developer must keep in mind is that messages cannot be modified, once published to a topic, as their memory is now shared with all nodelets subscribing to the topic.

\begin{listing}
    \inputminted{cpp}{\rootfolder/Chapters/Chapter5/Listings/object_detector_node.cpp}
    \caption{Nodelet instantiation using a C++ node.}
    \label{lst:detection-node-cpp}
\end{listing}

\begin{listing}
    \inputminted{xml}{\rootfolder/Chapters/Chapter5/Listings/ros_dnn.launch}
    \caption{Nodelet instantiation using a launch file.}
    \label{lst:detection-node-launch}
\end{listing}

In order to make up the object detection node, there are two approaches to launching the component nodelets that make up the node.
It is possible to implement a node in C++ in the usual manner, instantiate a nodelet Loader and use this to load each nodelet.
This is shown in~\cref{lst:detection-node-cpp}.
The alternative approach is to load a nodelet manager, and each nodelet, using a launch file.
This accomplishes the same result, but allows for changing the launch procedure without requiring a recompilation.
The approach is shown in~\cref{lst:detection-node-launch}.

The second advantage to using C++ is that we are able to use the implementation of the YOLO algorithm created by the OpenCV project.
As of recent versions of OpenCV, a wide range of object detection algorithms are available using the same programming interface.
This provides a level of flexibility, and also allows for the Cyborg project to benefit from improvements done by the OpenCV project, simply by upgrading to new versions of OpenCV in the future.

\begin{listing}
    \inputminted{cpp}{\rootfolder/Chapters/Chapter5/Listings/ObjectDetectorNodelet}
    \caption{ObjectDetectorNodelet class definition.}
    \label{lst:objectdetectornodelet}
\end{listing}

The full implementation can be found in the attachments, but we will give a run-through of the steps taken.
After the nodelet manager creates an instance of the class, the OnInit() function is run, which performs the steps normally found in the class constructor.
In the case of the ObjectDetectorNodelet, shown in~\cref{lst:objectdetectornodelet}, this function fetches parameters from the \acrshort{ros} parameter server, such as the names for the input and output topics as well as parameters required to load the neural network.
This includes a configuration file describing layers of the network, a file with weights for the network as described in~\cref{ch:background}, and so forth.

\begin{figure}[ht]
    \pimage{Figures/yolo}
    \caption[Screenshot of OpenCV object detection.]%
            {Screenshot of OpenCV object detection, courtesy of opencv.org.}
    \label{fig:opencv}
\end{figure}

Once all the required parameters are retrieved, the class loads the neural network and subscribes to the image and depth map topics published by zed\_ros\_wrapper.
The image and depth map topics are subscribed to using a SubscriberFilter, which ensures that we receive both images in the same callback function, so that we can easily process them together.
This also ensures that the received image and depth map have matching time stamps, by using the ExactSyncPolicy from the message\_filters library.
If the node is configured to run without subscribing to a depth map, this filtering step is not performed.
Finally, once topic subscriptions are configured, the output topics created so that the node can publish the object detection results.
We also set up dynamic reconfigure, so the user can change the classification threshold -- the minimum confidence required for a detected object to be published -- while the node is running.

\begin{listing}
    \inputminted{cpp}{\rootfolder/Chapters/Chapter5/Listings/Prediction}
    \caption{Prediction class definition.}
    \label{lst:prediction}
\end{listing}

On reception of an image, and optionally a depth map, the callback function \emph{camera\_cb()} is called.
Here, we check if there are any subscribers listening to the output of the node, and abort if there are not.
Then, the received image is converted to OpenCV format and fed through the neural network.
The returned output from the network, for each detected object, includes the object class label, the label confidence and coordinates for a bounding box around the object.
These are processed and a vector of Prediction objects, as shown in~\cref{lst:prediction}, are created to represent each detected object in the image.
If the depth map is available, this can also be included.

Finally, the predictions are drawn on the image, in the same way as shown in~\cref{fig:opencv}, and the image is published to the output image topic.
The predictions are also published as a list, in the form shown in~\cref{lst:prediction-msg}.

%----------------------------------------------------------------------------------------
%	SUBSECTION: PYTHON IMPLEMENTATION
%----------------------------------------------------------------------------------------

\subsection{Python implementation}

As described previously, the main advantage to using C++ over Python was believed to be in the ability to run the zed\_ros\_wrapper in the same process as the object detection algorithm.
In order to test this hypothesis, it was necessary to also implement the same functionality as a Python node.
This initially met a roadblock, as up until the new version of \acrshort{ros} released in May 2018 \acrshort{ros} used its own version of OpenCV.
The version of OpenCV included with \acrshort{ros} has typically lagged the latest OpenCV release by several months.
As the \acrshort{dnn} functionality in OpenCV has been rewritten very recently, this meant that this functionality was unavailable in \acrshort{ros} until recently.
For this reason, we initially implemented one version in Python without using OpenCV, and a second version after \acrshort{ros} released their new version.

\subsubsection{Initial version}

Initially, object detection was implemented in \acrshort{ros} using the PyYOLO library, which is a simple Python interface to the original C implementation of YOLO.
Just as the C implementation, the performance of this implementation is significantly higher than the OpenCV implementation described above.
This is hypothesized to be due to CUDA support in the C implementation, which is lacking in OpenCV.

The demonstration application created by \cite{Opheim2018} was modified to output the detection image and prediction list described earlier.
The program flow in the Python version of the object detection node is generally similar to the C++ implementation, but implementing in Python yields significantly more compact code.

\subsubsection{OpenCV version}

With the ability to use a custom compiled version of OpenCV with \acrshort{ros}, it became possible to use a version with support for cutting edge features, including the \acrfull{dnn} module.
While OpenCV has CUDA support for much of its functionality, the \acrshort{dnn} module is limited to \acrshort{gpu} acceleration through OpenCL which currently only supports Intel \acrshortpl{gpu}.
As the Cyborg project uses an Nvidia Jetson TX2 for graphics processing, this is a limitation.
As \acrshort{gpu} acceleration is key to achieving usable performance when running neural networks, backend support for this platform is key to making an OpenCV solution practical.
Even so, this makes it possible to perform an apples to apples performance comparison between a C++ and a Python implementation using OpenCV, which is interesting in itself.
Though this was only made possible shortly before the deadline for this report, it was interesting enough for the Cyborg project that this second Python implementation was created.

\end{document}
